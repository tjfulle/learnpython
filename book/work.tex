\chapter{Case study: interface design}
\label{turtlechap}

Code examples from this chapter are available from
\url{http://thinkpython.com/code/polygon.py}.


\section{TurtleWorld}
\label{turtleworld}
\index{TurtleWorld}
\index{Swampy}

To accompany this book, I have written a package called Swampy.
You can download Swampy from \url{http://thinkpython.com/swampy};
follow the instructions there to install Swampy on your system.

A {\bf package} is a collection of modules; one of the modules in
Swampy is {\tt TurtleWorld}, which provides a set of functions for
drawing lines by steering turtles around the screen.
\index{package}

If Swampy is installed as a package on your system, you can import
{\tt TurtleWorld} like this:

\begin{verbatim}
from swampy.TurtleWorld import *
\end{verbatim}

If you downloaded the Swampy modules but did not install them as a
package, you can either work in the directory that contains the Swampy
files, or add that directory to Python's search path.  Then you can import
{\tt TurtleWorld} like this:

\begin{verbatim}
from TurtleWorld import *
\end{verbatim}

The details of the installation process and setting Python's search
path depend on your system, so rather than include those details here,
I will try to maintain current information for several systems
at \url{http://thinkpython.com/swampy}

Create a file named {\tt mypolygon.py} and type in the following
code:

\begin{verbatim}
from swampy.TurtleWorld import *

world = TurtleWorld()
bob = Turtle()
print bob

wait_for_user()
\end{verbatim}
%
The first line imports everything from the {\tt TurtleWorld} module
in the {\tt swampy} package.
\index{import statement}
\index{statement!import}

The next lines create a TurtleWorld assigned to {\tt world} and
a Turtle assigned to {\tt bob}.  Printing {\tt bob} yields something
like:

\begin{verbatim}
<TurtleWorld.Turtle instance at 0xb7bfbf4c>
\end{verbatim}
%
This means that {\tt bob} refers to
an {\bf instance} of a Turtle
as defined in module {\tt TurtleWorld}.  In this context,
``instance'' means a member of a set;
this Turtle is one of the set of possible Turtles.
\index{instance}

\verb"wait_for_user" tells TurtleWorld to wait for the user
to do something, although in this case there's not much for
the user to do except close the window.

TurtleWorld provides several
turtle-steering functions: {\tt fd} and {\tt bk} for
forward and backward, and {\tt lt} and {\tt rt} for left and
right turns.  Also, each Turtle is holding a pen, which is
either down or up; if the pen is down, the Turtle leaves
a trail when it moves.  The functions {\tt pu} and {\tt pd}
stand for ``pen up'' and ``pen down.''

To draw a right angle, add these lines to the program
(after creating {\tt bob} and before calling \verb"wait_for_user"):

\begin{verbatim}
fd(bob, 100)
lt(bob)
fd(bob, 100)
\end{verbatim}
%
The first line tells {\tt bob} to take 100 steps
forward.  The second line tells him to turn left.

When you run this program, you should see {\tt bob} move east and then
north, leaving two line segments behind.

Now modify the program to draw a square.  Don't go on until
you've got it working!

%\newpage

\section{Simple repetition}
\label{repetition}
\index{repetition}

Chances are you wrote something like this (leaving out the code
that creates TurtleWorld and waits for the user):

\begin{verbatim}
fd(bob, 100)
lt(bob)

fd(bob, 100)
lt(bob)

fd(bob, 100)
lt(bob)

fd(bob, 100)
\end{verbatim}
%
We can do the same thing more concisely with a {\tt for} statement.
Add this example to {\tt mypolygon.py} and run it again:
\index{for loop}
\index{loop!for}
\index{statement!for}

\begin{verbatim}
for i in range(4):
    print 'Hello!'
\end{verbatim}
%
You should see something like this:

\begin{verbatim}
Hello!
Hello!
Hello!
Hello!
\end{verbatim}
%
This is the simplest use of the {\tt for} statement; we will see
more later.  But that should be enough to let you rewrite your
square-drawing program.  Don't go on until you do.

Here is a {\tt for} statement that draws a square:

\begin{verbatim}
for i in range(4):
    fd(bob, 100)
    lt(bob)
\end{verbatim}
%
The syntax of a {\tt for} statement is similar to a function
definition.  It has a header that ends with a colon and an indented
body.  The body can contain any number of statements.
\index{loop}

A {\tt for} statement is sometimes called a {\bf loop} because
the flow of execution runs through the body and then loops back
to the top.  In this case, it runs the body four times.

This version is actually a little different from the previous
square-drawing code because it makes another turn after
drawing the last side of the square.  The extra turn takes a little
more time, but it simplifies the code if we do the same thing
every time through the loop.  This version also has the effect
of leaving the turtle back in the starting position, facing in
the starting direction.

\section{Exercises}

The following is a series of exercises using TurtleWorld.  They
are meant to be fun, but they have a point, too.  While you are
working on them, think about what the point is.

The following sections have solutions to the exercises, so
don't look until you have finished (or at least tried).

\begin{enumerate}

\item Write a function called {\tt square} that takes a parameter
named {\tt t}, which is a turtle.  It should use the turtle to draw
a square.

Write a function call that passes {\tt bob} as an argument to
{\tt square}, and then run the program again.

\item Add another parameter, named {\tt length}, to {\tt square}.
Modify the body so length of the sides is {\tt length}, and then
modify the function call to provide a second argument.  Run the
program again.  Test your program with a range of values for {\tt
length}.

\item The functions {\tt lt} and {\tt rt} make 90-degree turns by
default, but you can provide a second argument that specifies the
number of degrees.  For example, {\tt lt(bob, 45)} turns {\tt bob} 45
degrees to the left.

Make a copy of {\tt square} and change the name to {\tt polygon}.  Add
another parameter named {\tt n} and modify the body so it draws an
n-sided regular polygon.  Hint: The exterior angles of an n-sided regular
polygon are $360/n$ degrees.
\index{polygon function}
\index{function!polygon}

\item Write a function called {\tt circle} that takes a turtle, {\tt t},
and radius, {\tt r}, as parameters and that draws an approximate circle
by invoking {\tt polygon} with an appropriate length and number of
sides.  Test your function with a range of values of {\tt r}.
\index{circle function}
\index{function!circle}

Hint: figure out the circumference of the circle and make sure that
{\tt length * n = circumference}.

Another hint: if {\tt bob} is too slow for you, you can speed
him up by changing {\tt bob.delay}, which is the time between moves,
in seconds.  {\tt bob.delay = 0.01} ought to get him moving.

% change this to world.delay

\item Make a more general version of {\tt circle} called {\tt arc}
that takes an additional parameter {\tt angle}, which determines
what fraction of a circle to draw.  {\tt angle} is in units of
degrees, so when {\tt angle=360}, {\tt arc} should draw a complete
circle.
\index{arc function}
\index{function!arc}

\end{enumerate}

\section{Encapsulation}

The first exercise asks you to put your square-drawing code
into a function definition and then call the function, passing
the turtle as a parameter.  Here is a solution:

\begin{verbatim}
def square(t):
    for i in range(4):
        fd(t, 100)
        lt(t)

square(bob)
\end{verbatim}
%
The innermost statements, {\tt fd} and {\tt lt} are
indented twice to show that they are inside the {\tt for} loop,
which is inside the function definition.  The next line,
{\tt square(bob)}, is flush with the left margin, so that is the
end of both the {\tt for} loop and the function definition.

Inside the function, {\tt t} refers to the same turtle {\tt bob}
refers to, so {\tt lt(t)} has the same effect as {\tt lt(bob)}.
So why not call the parameter {\tt bob}?  The idea is that {\tt t}
can be any turtle, not just {\tt bob}, so you could create
a second turtle and pass it as an argument to {\tt square}:

\begin{verbatim}
ray = Turtle()
square(ray)
\end{verbatim}
%
Wrapping a piece of code up in a function is called {\bf
encapsulation}.  One of the benefits of encapsulation is that it
attaches a name to the code, which serves as a kind of documentation.
Another advantage is that if you re-use the code, it is more concise
to call a function twice than to copy and paste the body!
\index{encapsulation}


\section{Generalization}

The next step is to add a {\tt length} parameter to {\tt square}.
Here is a solution:

\begin{verbatim}
def square(t, length):
    for i in range(4):
        fd(t, length)
        lt(t)

square(bob, 100)
\end{verbatim}
%
Adding a parameter to a function is called {\bf generalization}
because it makes the function more general: in the previous
version, the square is always the same size; in this version
it can be any size.
\index{generalization}

The next step is also a generalization.  Instead of drawing
squares, {\tt polygon} draws regular polygons with any number of
sides.  Here is a solution
:rule
\begin{verbatim}
def polygon(t, n, length):
    angle = 360.0 / n
    for i in range(n):
        fd(t, length)
        lt(t, angle)

polygon(bob, 7, 70)
\end{verbatim}
%
This draws a 7-sided polygon with side length 70.  If you have
more than a few numeric arguments, it is easy to forget what they
are, or what order they should be in.  It is legal, and sometimes
helpful, to include the names of the parameters in the argument
list:

\begin{verbatim}
polygon(bob, n=7, length=70)
\end{verbatim}
%
These are called {\bf keyword arguments} because they include
the parameter names as ``keywords'' (not to be confused with
Python keywords like {\tt while} and {\tt def}).
\index{keyword argument}
\index{argument!keyword}

This syntax makes the program more readable.  It is also a reminder
about how arguments and parameters work: when you call a function, the
arguments are assigned to the parameters.


\section{Interface design}

The next step is to write {\tt circle}, which takes a radius,
{\tt r}, as a parameter.  Here is a simple solution that uses
{\tt polygon} to draw a 50-sided polygon:

\begin{verbatim}
def circle(t, r):
    circumference = 2 * math.pi * r
    n = 50
    length = circumference / n
    polygon(t, n, length)
\end{verbatim}
%
The first line computes the circumference of a circle with radius
{\tt r} using the formula $2 \pi r$.  Since we use {\tt math.pi}, we
have to import {\tt math}.  By convention, {\tt import} statements
are usually at the beginning of the script.

{\tt n} is the number of line segments in our approximation of a circle,
so {\tt length} is the length of each segment.  Thus, {\tt polygon}
draws a 50-sides polygon that approximates a circle with radius {\tt r}.

One limitation of this solution is that {\tt n} is a constant, which
means that for very big circles, the line segments are too long, and
for small circles, we waste time drawing very small segments.  One
solution would be to generalize the function by taking {\tt n} as
a parameter.  This would give the user (whoever calls {\tt circle})
more control, but the interface would be less clean.
\index{interface}

The {\bf interface} of a function is a summary of how it is used: what
are the parameters?  What does the function do?  And what is the return
value?  An interface is ``clean'' if it is ``as simple as
possible, but not simpler. (Einstein)''
\index{Einstein, Albert}

In this example, {\tt r} belongs in the interface because it
specifies the circle to be drawn.  {\tt n} is less appropriate
because it pertains to the details of {\em how} the circle should
be rendered.

Rather than clutter up the interface, it is better
to choose an appropriate value of {\tt n}
depending on {\tt circumference}:

\begin{verbatim}
def circle(t, r):
    circumference = 2 * math.pi * r
    n = int(circumference / 3) + 1
    length = circumference / n
    polygon(t, n, length)
\end{verbatim}
%
Now the number of segments is (approximately) {\tt circumference/3},
so the length of each segment is (approximately) 3, which is small
enough that the circles look good, but big enough to be efficient,
and appropriate for any size circle.


\section{Refactoring}
\label{refactoring}
\index{refactoring}

When I wrote {\tt circle}, I was able to re-use {\tt polygon}
because a many-sided polygon is a good approximation of a circle.
But {\tt arc} is not as cooperative; we can't use {\tt polygon}
or {\tt circle} to draw an arc.

One alternative is to start with a copy
of {\tt polygon} and transform it into {\tt arc}.  The result
might look like this:

\begin{verbatim}
def arc(t, r, angle):
    arc_length = 2 * math.pi * r * angle / 360
    n = int(arc_length / 3) + 1
    step_length = arc_length / n
    step_angle = float(angle) / n

    for i in range(n):
        fd(t, step_length)
        lt(t, step_angle)
\end{verbatim}
%
The second half of this function looks like {\tt polygon}, but we
can't re-use {\tt polygon} without changing the interface.  We could
generalize {\tt polygon} to take an angle as a third argument,
but then {\tt polygon} would no longer be an appropriate name!
Instead, let's call the more general function {\tt polyline}:

\begin{verbatim}
def polyline(t, n, length, angle):
    for i in range(n):
        fd(t, length)
        lt(t, angle)
\end{verbatim}
%
Now we can rewrite {\tt polygon} and {\tt arc} to use {\tt polyline}:

\begin{verbatim}
def polygon(t, n, length):
    angle = 360.0 / n
    polyline(t, n, length, angle)

def arc(t, r, angle):
    arc_length = 2 * math.pi * r * angle / 360
    n = int(arc_length / 3) + 1
    step_length = arc_length / n
    step_angle = float(angle) / n
    polyline(t, n, step_length, step_angle)
\end{verbatim}
%
Finally, we can rewrite {\tt circle} to use {\tt arc}:

\begin{verbatim}
def circle(t, r):
    arc(t, r, 360)
\end{verbatim}
%
This process---rearranging a program to improve function
interfaces and facilitate code re-use---is called {\bf refactoring}.
In this case, we noticed that there was similar code in {\tt arc} and
{\tt polygon}, so we ``factored it out'' into {\tt polyline}.
\index{refactoring}

If we had planned ahead, we might have written {\tt polyline} first
and avoided refactoring, but often you don't know enough at the
beginning of a project to design all the interfaces.  Once you start
coding, you understand the problem better.  Sometimes refactoring is a
sign that you have learned something.


\section{A development plan}
\index{development plan!encapsulation and generalization}

A {\bf development plan} is a process for writing programs.
The process we used
in this case study is ``encapsulation and
generalization.''  The steps of this process are:

\begin{enumerate}

\item Start by writing a small program with no function definitions.

\item Once you get the program working, encapsulate it in a function
and give it a name.

\item Generalize the function by adding appropriate parameters.

\item Repeat steps 1--3 until you have a set of working functions.
Copy and paste working code to avoid retyping (and re-debugging).

\item Look for opportunities to improve the program by refactoring.
For example, if you have similar code in several places, consider
factoring it into an appropriately general function.

\end{enumerate}

This process has some drawbacks---we will see alternatives later---but
it can be useful if you don't know ahead of time how to divide the
program into functions.  This approach lets you design as you go
along.


\section{docstring}
\label{docstring}
\index{docstring}

A {\bf docstring} is a string at the beginning of a function that
explains the interface (``doc'' is short for ``documentation'').  Here
is an example:

\begin{verbatim}
def polyline(t, n, length, angle):
    """Draws n line segments with the given length and
    angle (in degrees) between them.  t is a turtle.
    """
    for i in range(n):
        fd(t, length)
        lt(t, angle)
\end{verbatim}
%
This docstring is a triple-quoted string, also known
as a multiline string because the triple quotes allow the string
to span more than one line.
\index{quotation mark}
\index{triple-quoted string}
\index{string!triple-quoted}
\index{multiline string}
\index{string!multiline}

It is terse, but it contains the essential information
someone would need to use this function.  It explains concisely what
the function does (without getting into the details of how it does
it).  It explains what effect each parameter has on the behavior of
the function and what type each parameter should be (if it is not
obvious).

Writing this kind of documentation is an important part of interface
design.  A well-designed interface should be simple to explain;
if you are having a hard time explaining one of your functions,
that might be a sign that the interface could be improved.


\section{Debugging}
\index{debugging}
\index{interface}

An interface is like a contract between a function and a caller.
The caller agrees to provide certain parameters and the function
agrees to do certain work.

For example, {\tt polyline} requires four arguments: {\tt t} has to be
a Turtle; {\tt n} is the number of line segments, so it has to be an
integer; {\tt length} should be a positive number; and {\tt
  angle} has to be a number, which is understood to be in degrees.

These requirements are called {\bf preconditions} because they
are supposed to be true before the function starts executing.
Conversely, conditions at the end of the function are
{\bf postconditions}.  Postconditions include the intended
effect of the function (like drawing line segments) and any
side effects (like moving the Turtle or making other changes
in the World).
\index{precondition}
\index{postcondition}

Preconditions are the responsibility of the caller.  If the caller
violates a (properly documented!) precondition and the function
doesn't work correctly, the bug is in the caller, not the function.

% Removing this because we haven't seen conditionals yet!
%However, for purposes of debugging it is often a good idea for
%functions to check their preconditions rather than assume they are
%true.  If every function checks its preconditions before starting,
%then if something goes wrong, you will know which function to blame.


\section{Glossary}

\begin{description}

\item[instance:] A member of a set.  The TurtleWorld in this
chapter is a member of the set of TurtleWorlds.
\index{instance}

\item[loop:] A part of a program that can execute repeatedly.
\index{loop}

\item[encapsulation:] The process of transforming a sequence of
statements into a function definition.
\index{encapsulation}

\item[generalization:] The process of replacing something
unnecessarily specific (like a number) with something appropriately
general (like a variable or parameter).
\index{generalization}

\item[keyword argument:] An argument that includes the name of
the parameter as a ``keyword.''
\index{keyword argument}
\index{argument!keyword}

\item[interface:] A description of how to use a function, including
the name and descriptions of the arguments and return value.
\index{interface}

\item[refactoring:] The process of modifying a working program to
  improve function interfaces and other qualities of the code.
\index{refactoring}

\item[development plan:] A process for writing programs.
\index{development plan}

\item[docstring:]  A string that appears in a function definition
to document the function's interface.
\index{docstring}

\item[precondition:] A requirement that should be satisfied by
the caller before a function starts.
\index{precondition}

\item[postcondition:] A requirement that should be satisfied by
the function before it ends.
\index{precondition}

\end{description}


\section{Exercises}

\begin{exercise}

Download the code in this chapter from
\url{http://thinkpython.com/code/polygon.py}.

\begin{enumerate}

\item Write appropriate docstrings for {\tt polygon}, {\tt arc} and
{\tt circle}.
\index{stack diagram}

\item Draw a stack diagram that shows the state of the program
while executing {\tt circle(bob, radius)}.  You can do the
arithmetic by hand or add {\tt print} statements to the code.

\item The version of {\tt arc} in Section~\ref{refactoring} is not
very accurate because the linear approximation of the
circle is always outside the true circle.  As a result,
the turtle ends up a few units away from the correct
destination. My solution shows a way to reduce
the effect of this error.  Read the code and see if it makes
sense to you.  If you draw a diagram, you might see how it works.

\end{enumerate}

\end{exercise}

\begin{figure}
\centerline
{\includegraphics[scale=0.8]{figs/flowers.pdf}}
\caption{Turtle flowers.}
\label{fig.flowers}
\end{figure}

\begin{exercise}
\index{flower}

Write an appropriately general set of functions that
can draw flowers as in Figure~\ref{fig.flowers}.

Solution: \url{http://thinkpython.com/code/flower.py},
also requires \url{http://thinkpython.com/code/polygon.py}.

\end{exercise}

\begin{figure}
\centerline
{\includegraphics[scale=0.8]{figs/pies.pdf}}
\caption{Turtle pies.}
\label{fig.pies}
\end{figure}


\begin{exercise}
\index{pie}

Write an appropriately general set of functions that
can draw shapes as in Figure~\ref{fig.pies}.

Solution: \url{http://thinkpython.com/code/pie.py}.

\end{exercise}

\begin{exercise}
\index{alphabet}
\index{turtle typewriter}
\index{typewriter, turtle}

The letters of the alphabet can be constructed from a moderate number
of basic elements, like vertical and horizontal lines and a few
curves.  Design a font that can be drawn with a minimal number of
basic elements and then write functions that draw letters of the
alphabet.

You should write one function for each letter, with names
\verb"draw_a", \verb"draw_b", etc., and put your functions
in a file named {\tt letters.py}.  You can download a
``turtle typewriter'' from \url{http://thinkpython.com/code/typewriter.py}
to help you test your code.

Solution: \url{http://thinkpython.com/code/letters.py}, also requires
\url{http://thinkpython.com/code/polygon.py}.

\end{exercise}

\begin{exercise}

Read about spirals at \url{http://en.wikipedia.org/wiki/Spiral}; then
write a program that draws an Archimedian spiral (or one of the other
kinds).  Solution: \url{http://thinkpython.com/code/spiral.py}.
\index{spiral}
\index{Archimedian spiral}

\end{exercise}


\chapter{Case study: word play}

\section{Reading word lists}
\label{wordlist}

For the exercises in this chapter we need a list of English words.
There are lots of word lists available on the Web, but the one most
suitable for our purpose is one of the word lists collected and
contributed to the public domain by Grady Ward as part of the Moby
lexicon project (see \url{http://wikipedia.org/wiki/Moby_Project}).  It
is a list of 113,809 official crosswords; that is, words that are
considered valid in crossword puzzles and other word games.  In the
Moby collection, the filename is {\tt 113809of.fic}; you can download
a copy, with the simpler name {\tt words.txt}, from
\url{http://thinkpython.com/code/words.txt}.
\index{Moby Project}
\index{crosswords}

This file is in plain text, so you can open it with a text
editor, but you can also read it from Python.  The built-in
function {\tt open} takes the name of the file as a parameter
and returns a {\bf file object} you can use to read the file.
\index{open function}
\index{function!open}
\index{plain text}
\index{text!plain}
\index{object!file}
\index{file object}

\begin{verbatim}
>>> fin = open('words.txt')
>>> print fin
<open file 'words.txt', mode 'r' at 0xb7f4b380>
\end{verbatim}
%
{\tt fin} is a common name for a file object used for
input.  Mode \verb"'r'" indicates that this file is open for
reading (as opposed to \verb"'w'" for writing).
\index{readline method}
\index{method!readline}

The file object provides several methods for reading, including
{\tt readline}, which reads characters from the file
until it gets to a newline and returns the result as a
string:

\begin{verbatim}
>>> fin.readline()
'aa\r\n'
\end{verbatim}
%
The first word in this particular list is ``aa,'' which is a kind of
lava.  The sequence \verb"\r\n" represents two whitespace characters,
a carriage return and a newline, that separate this word from the
next.

The file object keeps track of where it is in the file, so
if you call {\tt readline} again, you get the next word:

\begin{verbatim}
>>> fin.readline()
'aah\r\n'
\end{verbatim}
%
The next word is ``aah,'' which is a perfectly legitimate
word, so stop looking at me like that.
Or, if it's the whitespace that's bothering you,
we can get rid of it with the string method {\tt strip}:
\index{strip method}
\index{method!strip}

\begin{verbatim}
>>> line = fin.readline()
>>> word = line.strip()
>>> print word
aahed
\end{verbatim}
%
You can also use a file object as part of a {\tt for} loop.
This program reads {\tt words.txt} and prints each word, one
per line:
\index{open function}
\index{function!open}

\begin{verbatim}
fin = open('words.txt')
for line in fin:
    word = line.strip()
    print word
\end{verbatim}
%

\begin{exercise}

Write a program that reads {\tt words.txt} and prints only the
words with more than 20 characters (not counting whitespace).
\index{whitespace}

\end{exercise}


\section{Exercises}

There are solutions to these exercises in the next section.
You should at least attempt each one before you read the solutions.

\begin{exercise}

In 1939 Ernest Vincent Wright published a 50,000 word novel called
{\em Gadsby} that does not contain the letter ``e.''  Since ``e'' is
the most common letter in English, that's not easy to do.

In fact, it is difficult to construct a solitary thought without using
that most common symbol.  It is slow going at first, but with caution
and hours of training you can gradually gain facility.

All right, I'll stop now.

Write a function called \verb"has_no_e" that returns {\tt True} if
the given word doesn't have the letter ``e'' in it.

Modify your program from the previous section to print only the words
that have no ``e'' and compute the percentage of the words in the list
have no ``e.''
\index{lipogram}

\end{exercise}


\begin{exercise}

Write a function named {\tt avoids}
that takes a word and a string of forbidden letters, and
that returns {\tt True} if the word doesn't use any of the forbidden
letters.

Modify your program to prompt the user to enter a string
of forbidden letters and then print the number of words that
don't contain any of them.
Can you find a combination of 5 forbidden letters that
excludes the smallest number of words?

\end{exercise}



\begin{exercise}

Write a function named \verb"uses_only" that takes a word and a
string of letters, and that returns {\tt True} if the word contains
only letters in the list.  Can you make a sentence using only the
letters {\tt acefhlo}?  Other than ``Hoe alfalfa?''

\end{exercise}


\begin{exercise}

Write a function named \verb"uses_all" that takes a word and a
string of required letters, and that returns {\tt True} if the word
uses all the required letters at least once.  How many words are there
that use all the vowels {\tt aeiou}?  How about {\tt aeiouy}?

\end{exercise}


\begin{exercise}

Write a function called \verb"is_abecedarian" that returns
{\tt True} if the letters in a word appear in alphabetical order
(double letters are ok).
How many abecedarian words are there?

\index{abecedarian}

\end{exercise}


%\begin{exercise}
%\label{palindrome}
%A palindrome is a word that reads the same
%forward and backward, like ``rotator'' and ``noon.''
%Write a boolean function named \verb"is_palindrome" that
%takes a string as a parameter and returns {\tt True} if it is
%a palindrome.

%Modify your program from the previous section to print all
%of the palindromes in the word list and then print the total
%number of palindromes.
%\end{exercise}



\section{Search}
\index{search pattern}
\index{pattern!search}

All of the exercises in the previous section have something
in common; they can be solved with the search pattern we saw
in Section~\ref{find}.  The simplest example is:

\begin{verbatim}
def has_no_e(word):
    for letter in word:
        if letter == 'e':
            return False
    return True
\end{verbatim}
%
The {\tt for} loop traverses the characters in {\tt word}.  If we find
the letter ``e'', we can immediately return {\tt False}; otherwise we
have to go to the next letter.  If we exit the loop normally, that
means we didn't find an ``e'', so we return {\tt True}.
\index{traversal}

% Removing this because we haven't seen the in operator yet.
%\index{in operator}
%\index{operator!in}

%You could write this function more concisely using the {\tt in}
%operator, but I started with this version because it
%demonstrates the logic of the search pattern.
\index{generalization}

{\tt avoids} is a more general version of \verb"has_no_e" but it
has the same structure:

\begin{verbatim}
def avoids(word, forbidden):
    for letter in word:
        if letter in forbidden:
            return False
    return True
\end{verbatim}
%
We can return {\tt False} as soon as we find a forbidden letter;
if we get to the end of the loop, we return {\tt True}.

\verb"uses_only" is similar except that the sense of the condition
is reversed:

\begin{verbatim}
def uses_only(word, available):
    for letter in word:
        if letter not in available:
            return False
    return True
\end{verbatim}
%
Instead of a list of forbidden letters, we have a list of available
letters.  If we find a letter in {\tt word} that is not in
{\tt available}, we can return {\tt False}.

\verb"uses_all" is similar except that we reverse the role
of the word and the string of letters:

\begin{verbatim}
def uses_all(word, required):
    for letter in required:
        if letter not in word:
            return False
    return True
\end{verbatim}
%
Instead of traversing the letters in {\tt word}, the loop
traverses the required letters.  If any of the required letters
do not appear in the word, we can return {\tt False}.
\index{traversal}

If you were really thinking like a computer scientist, you would
have recognized that \verb"uses_all" was an instance of a
previously-solved problem, and you would have written:

\begin{verbatim}
def uses_all(word, required):
    return uses_only(required, word)
\end{verbatim}
%
This is an example of a program development method called {\bf problem
recognition}, which means that you recognize the problem you are
working on as an instance of a previously-solved problem, and apply a
previously-developed solution.
\index{problem recognition}
\index{development plan!problem recognition}


\section{Looping with indices}
\index{looping!with indices}
\index{index!looping with}

I wrote the functions in the previous section with {\tt for}
loops because I only needed the characters in the strings; I didn't
have to do anything with the indices.

For \verb"is_abecedarian" we have to compare adjacent letters,
which is a little tricky with a {\tt for} loop:

\begin{verbatim}
def is_abecedarian(word):
    previous = word[0]
    for c in word:
        if c < previous:
            return False
        previous = c
    return True
\end{verbatim}


An alternative is to
use recursion:

\begin{verbatim}
def is_abecedarian(word):
    if len(word) <= 1:
        return True
    if word[0] > word[1]:
        return False
    return is_abecedarian(word[1:])
\end{verbatim}

Another option is to use a {\tt while} loop:

\begin{verbatim}
def is_abecedarian(word):
    i = 0
    while i < len(word)-1:
        if word[i+1] < word[i]:
            return False
        i = i+1
    return True
\end{verbatim}
%
The loop starts at {\tt i=0} and ends when {\tt i=len(word)-1}.  Each
time through the loop, it compares the $i$th character (which you can
think of as the current character) to the $i+1$th character (which you
can think of as the next).

If the next character is less than (alphabetically before) the current
one, then we have discovered a break in the abecedarian trend, and
we return {\tt False}.

If we get to the end of the loop without finding a fault, then the
word passes the test.  To convince yourself that the loop ends
correctly, consider an example like \verb"'flossy'".  The
length of the word is 6, so
the last time the loop runs is when {\tt i} is 4, which is the
index of the second-to-last character.  On the last iteration,
it compares the second-to-last character to the last, which is
what we want.
\index{palindrome}

Here is a version of \verb"is_palindrome" (see
Exercise~\ref{palindrome}) that uses two indices; one starts at the
beginning and goes up; the other starts at the end and goes down.

\begin{verbatim}
def is_palindrome(word):
    i = 0
    j = len(word)-1

    while i<j:
        if word[i] != word[j]:
            return False
        i = i+1
        j = j-1

    return True
\end{verbatim}

Or, if you noticed that this is an instance of a previously-solved
problem, you might have written:

\begin{verbatim}
def is_palindrome(word):
    return is_reverse(word, word)
\end{verbatim}
\index{problem recognition}
\index{development plan!problem recognition}

Assuming you did Exercise~\ref{isreverse}.


\section{Debugging}
\index{debugging}
\index{testing!is hard}
\index{program testing}

Testing programs is hard.  The functions in this chapter are
relatively easy to test because you can check the results by hand.
Even so, it is somewhere between difficult and impossible to choose a
set of words that test for all possible errors.

Taking \verb"has_no_e" as an example, there are two obvious
cases to check: words that have an 'e' should return {\tt False};
words that don't should return {\tt True}.  You should have no
trouble coming up with one of each.

Within each case, there are some less obvious subcases.  Among the
words that have an ``e,'' you should test words with an ``e'' at the
beginning, the end, and somewhere in the middle.  You should test long
words, short words, and very short words, like the empty string.  The
empty string is an example of a {\bf special case}, which is one of
the non-obvious cases where errors often lurk.
\index{special case}

In addition to the test cases you generate, you can also test
your program with a word list like {\tt words.txt}.  By scanning
the output, you might be able to catch errors, but be careful:
you might catch one kind of error (words that should not be
included, but are) and not another (words that should be included,
but aren't).

In general, testing can help you find bugs, but it is not easy to
generate a good set of test cases, and even if you do, you can't
be sure your program is correct.
\index{testing!and absence of bugs}

According to a legendary computer scientist:

\begin{quote}
Program testing can be used to show the presence of bugs, but never to
show their absence!

--- Edsger W. Dijkstra
\end{quote}
\index{Dijkstra, Edsger}


\section{Glossary}

\begin{description}

\item[file object:] A value that represents an open file.
\index{file object}
\index{object!file}

\item[problem recognition:] A way of solving a problem by
expressing it as an instance of a previously-solved problem.
\index{problem recognition}

\item[special case:] A test case that is atypical or non-obvious
(and less likely to be handled correctly).
\index{special case}

\end{description}


\section{Exercises}

\begin{exercise}
\index{Car Talk}
\index{Puzzler}
\index{double letters}

This question is based on a Puzzler that was broadcast on the radio
program {\em Car Talk}
(\url{http://www.cartalk.com/content/puzzlers}):

\begin{quote}
Give me a word with three consecutive double letters. I'll give you a
couple of words that almost qualify, but don't. For example, the word
committee, c-o-m-m-i-t-t-e-e. It would be great except for the `i' that
sneaks in there. Or Mississippi: M-i-s-s-i-s-s-i-p-p-i. If you could
take out those i's it would work. But there is a word that has three
consecutive pairs of letters and to the best of my knowledge this may
be the only word. Of course there are probably 500 more but I can only
think of one. What is the word?
\end{quote}

Write a program to find it.  Solution: \url{http://thinkpython.com/code/cartalk1.py}.

\end{exercise}


\begin{exercise}
Here's another {\em Car Talk}
Puzzler (\url{http://www.cartalk.com/content/puzzlers}):
\index{Car Talk}
\index{Puzzler}
\index{odometer}
\index{palindrome}

\begin{quote}
``I was driving on the highway the other day and I happened to
notice my odometer. Like most odometers, it shows six digits,
in whole miles only. So, if my car had 300,000
miles, for example, I'd see 3-0-0-0-0-0.

``Now, what I saw that day was very interesting. I noticed that the
last 4 digits were palindromic; that is, they read the same forward as
backward. For example, 5-4-4-5 is a palindrome, so my odometer
could have read 3-1-5-4-4-5.

``One mile later, the last 5 numbers were palindromic. For example, it
could have read 3-6-5-4-5-6.  One mile after that, the middle 4 out of
6 numbers were palindromic.  And you ready for this? One mile later,
all 6 were palindromic!

``The question is, what was on the odometer when I first looked?''
\end{quote}

Write a Python program that tests all the six-digit numbers and prints
any numbers that satisfy these requirements.
Solution: \url{http://thinkpython.com/code/cartalk2.py}.

\end{exercise}


\begin{exercise}
Here's another {\em Car Talk} Puzzler you can solve with a
search (\url{http://www.cartalk.com/content/puzzlers}):
\index{Car Talk}
\index{Puzzler}
\index{palindrome}

\begin{quote}
``Recently I had a visit with my mom and we realized that
the two digits that make up my age when reversed resulted in her
age. For example, if she's 73, I'm 37. We wondered how often this has
happened over the years but we got sidetracked with other topics and
we never came up with an answer.

``When I got home I figured out that the digits of our ages have been
reversible six times so far. I also figured out that if we're lucky it
would happen again in a few years, and if we're really lucky it would
happen one more time after that. In other words, it would have
happened 8 times over all. So the question is, how old am I now?''

\end{quote}

Write a Python program that searches for solutions to this Puzzler.
Hint: you might find the string method {\tt zfill} useful.

Solution: \url{http://thinkpython.com/code/cartalk3.py}.

\end{exercise}



\chapter{Case study: data structure selection}

\section{Word frequency analysis}
\label{analysis}

As usual, you should at least attempt the following exercises
before you read my solutions.

\begin{exercise}

Write a program that reads a file, breaks each line into
words, strips whitespace and punctuation from the words, and
converts them to lowercase.
\index{string module}
\index{module!string}

Hint: The {\tt string} module provides strings named {\tt whitespace},
which contains space, tab, newline, etc., and {\tt
  punctuation} which contains the punctuation characters.  Let's see
if we can make Python swear:

\begin{verbatim}
>>> import string
>>> print string.punctuation
!"#$%&'()*+,-./:;<=>?@[\]^_`{|}~
\end{verbatim}
%
Also, you might consider using the string methods {\tt strip},
{\tt replace} and {\tt translate}.
\index{strip method}
\index{method!strip}
\index{replace method}
\index{method!replace}
\index{translate method}
\index{method!translate}

\end{exercise}


\begin{exercise}
\index{Project Gutenberg}

Go to Project Gutenberg (\url{http://gutenberg.org}) and download
your favorite out-of-copyright book in plain text format.
\index{plain text}
\index{text!plain}

Modify your program from the previous exercise to read the book
you downloaded, skip over the header information at the beginning
of the file, and process the rest of the words as before.

Then modify the program to count the total number of words in
the book, and the number of times each word is used.
\index{word frequency}
\index{frequency!word}

Print the number of different words used in the book.  Compare
different books by different authors, written in different eras.
Which author uses the most extensive vocabulary?
\end{exercise}


\begin{exercise}

Modify the program from the previous exercise to print the
20 most frequently-used words in the book.

\end{exercise}


\begin{exercise}

Modify the previous program to read a word list (see
Section~\ref{wordlist}) and then print all the words in the book that
are not in the word list.  How many of them are typos?  How many of
them are common words that {\em should} be in the word list, and how
many of them are really obscure?

\end{exercise}


\section{Random numbers}
\index{random number}
\index{number, random}
\index{deterministic}
\index{pseudorandom}

Given the same inputs, most computer programs generate the same
outputs every time, so they are said to be {\bf deterministic}.
Determinism is usually a good thing, since we expect the same
calculation to yield the same result.  For some applications, though,
we want the computer to be unpredictable.  Games are an obvious
example, but there are more.

Making a program truly nondeterministic turns out to be not so easy,
but there are ways to make it at least seem nondeterministic.  One of
them is to use algorithms that generate {\bf pseudorandom} numbers.
Pseudorandom numbers are not truly random because they are generated
by a deterministic computation, but just by looking at the numbers it
is all but impossible to distinguish them from random.
\index{random module}
\index{module!random}

The {\tt random} module provides functions that generate
pseudorandom numbers (which I will simply call ``random'' from
here on).
\index{random function}
\index{function!random}

The function {\tt random} returns a random float
between 0.0 and 1.0 (including 0.0 but not 1.0).  Each time you
call {\tt random}, you get the next number in a long series.  To see a
sample, run this loop:

\begin{verbatim}
import random

for i in range(10):
    x = random.random()
    print x
\end{verbatim}
%
The function {\tt randint} takes parameters {\tt low} and
{\tt high} and returns an integer between {\tt low} and
{\tt high} (including both).
\index{randint function}
\index{function!randint}

\begin{verbatim}
>>> random.randint(5, 10)
5
>>> random.randint(5, 10)
9
\end{verbatim}
%
To choose an element from a sequence at random, you can use
{\tt choice}:
\index{choice function}
\index{function!choice}

\begin{verbatim}
>>> t = [1, 2, 3]
>>> random.choice(t)
2
>>> random.choice(t)
3
\end{verbatim}
%
The {\tt random} module also provides functions to generate
random values from continuous distributions including
Gaussian, exponential, gamma, and a few more.

\begin{exercise}
\index{histogram!random choice}

Write a function named \verb"choose_from_hist" that takes
a histogram as defined in Section~\ref{histogram} and returns a
random value from the histogram, chosen with probability
in proportion to frequency.  For example, for this histogram:

\begin{verbatim}
>>> t = ['a', 'a', 'b']
>>> hist = histogram(t)
>>> print hist
{'a': 2, 'b': 1}
\end{verbatim}
%
your function should return {\tt 'a'} with probability $2/3$ and {\tt 'b'}
with probability $1/3$.
\end{exercise}


\section{Word histogram}

You should attempt the previous exercises before you go on.
You can download my solution from
 \url{http://thinkpython.com/code/analyze_book.py}.  You will
also need \url{http://thinkpython.com/code/emma.txt}.

Here is a program that reads a file and builds a histogram of the
words in the file:
\index{histogram!word frequencies}

\begin{verbatim}
import string

def process_file(filename):
    hist = dict()
    fp = open(filename)
    for line in fp:
        process_line(line, hist)
    return hist

def process_line(line, hist):
    line = line.replace('-', ' ')

    for word in line.split():
        word = word.strip(string.punctuation + string.whitespace)
        word = word.lower()

        hist[word] = hist.get(word, 0) + 1

hist = process_file('emma.txt')
\end{verbatim}
%
This program reads {\tt emma.txt}, which contains the text of {\em
  Emma} by Jane Austen.
\index{Austin, Jane}

\verb"process_file" loops through the lines of the file,
passing them one at a time to \verb"process_line".  The histogram
{\tt hist} is being used as an accumulator.
\index{accumulator!histogram}
\index{traversal}

\verb"process_line" uses the string method {\tt replace} to replace
hyphens with spaces before using {\tt split} to break the line into a
list of strings.  It traverses the list of words and uses {\tt strip}
and {\tt lower} to remove punctuation and convert to lower case.  (It
is a shorthand to say that strings are ``converted;'' remember that
string are immutable, so methods like {\tt strip} and {\tt lower}
return new strings.)

Finally, \verb"process_line" updates the histogram by creating a new
item or incrementing an existing one.
\index{update!histogram}

To count the total number of words in the file, we can add up
the frequencies in the histogram:

\begin{verbatim}
def total_words(hist):
    return sum(hist.values())
\end{verbatim}
%
The number of different words is just the number of items in
the dictionary:

\begin{verbatim}
def different_words(hist):
    return len(hist)
\end{verbatim}
%
Here is some code to print the results:

\begin{verbatim}
print 'Total number of words:', total_words(hist)
print 'Number of different words:', different_words(hist)
\end{verbatim}
%
And the results:

\begin{verbatim}
Total number of words: 161080
Number of different words: 7214
\end{verbatim}
%

\section{Most common words}
\index{DSU pattern}
\index{pattern!DSU}

To find the most common words, we can apply the DSU pattern;
\verb"most_common" takes a histogram and returns a list of
word-frequency tuples, sorted in reverse order by frequency:

\begin{verbatim}
def most_common(hist):
    t = []
    for key, value in hist.items():
        t.append((value, key))

    t.sort(reverse=True)
    return t
\end{verbatim}
%
Here is a loop that prints the ten most common words:

\begin{verbatim}
t = most_common(hist)
print 'The most common words are:'
for freq, word in t[0:10]:
    print word, '\t', freq
\end{verbatim}
%
And here are the results from {\em Emma}:

\begin{verbatim}
The most common words are:
to 	5242
the 	5205
and 	4897
of 	4295
i 	3191
a 	3130
it 	2529
her 	2483
was 	2400
she 	2364
\end{verbatim}
%

\section{Optional parameters}
\index{optional parameter}
\index{parameter!optional}

We have seen built-in functions and methods that take a variable
number of arguments.  It is possible to write user-defined functions
with optional arguments, too.  For example, here is a function that
prints the most common words in a histogram

\begin{verbatim}
def print_most_common(hist, num=10):
    t = most_common(hist)
    print 'The most common words are:'
    for freq, word in t[:num]:
        print word, '\t', freq
\end{verbatim}

The first parameter is required; the second is optional.
The {\bf default value} of {\tt num} is 10.
\index{default value}
\index{value!default}

If you only provide one argument:

\begin{verbatim}
print_most_common(hist)
\end{verbatim}

{\tt num} gets the default value.  If you provide two arguments:

\begin{verbatim}
print_most_common(hist, 20)
\end{verbatim}

{\tt num} gets the value of the argument instead.  In other
words, the optional argument {\bf overrides} the default value.
\index{override}

If a function has both required and optional parameters, all
the required parameters have to come first, followed by the
optional ones.


\section{Dictionary subtraction}
\index{dictionary!subtraction}
\index{subtraction!dictionary}

Finding the words from the book that are not in the word list
from {\tt words.txt} is a problem you might recognize as set
subtraction; that is, we want to find all the words from one
set (the words in the book) that are not in another set (the
words in the list).

{\tt subtract} takes dictionaries {\tt d1} and {\tt d2} and returns a
new dictionary that contains all the keys from {\tt d1} that are not
in {\tt d2}.  Since we don't really care about the values, we
set them all to None.

\begin{verbatim}
def subtract(d1, d2):
    res = dict()
    for key in d1:
        if key not in d2:
            res[key] = None
    return res
\end{verbatim}
%
To find the words in the book that are not in {\tt words.txt},
we can use \verb"process_file" to build a histogram for
{\tt words.txt}, and then subtract:

\begin{verbatim}
words = process_file('words.txt')
diff = subtract(hist, words)

print "The words in the book that aren't in the word list are:"
for word in diff.keys():
    print word,
\end{verbatim}
%
Here are some of the results from {\em Emma}:

\begin{verbatim}
The words in the book that aren't in the word list are:
 rencontre jane's blanche woodhouses disingenuousness
friend's venice apartment ...
\end{verbatim}
%
Some of these words are names and possessives.  Others, like
``rencontre,'' are no longer in common use.  But a few are common
words that should really be in the list!

\begin{exercise}
\index{set}
\index{type!set}

Python provides a data structure called {\tt set} that provides many
common set operations.  Read the documentation at
\url{http://docs.python.org/2/library/stdtypes.html#types-set} and
write a program that uses set subtraction to find words in the book
that are not in the word list.  Solution:
\url{http://thinkpython.com/code/analyze_book2.py}.

\end{exercise}


\section{Random words}
\label{randomwords}
\index{histogram!random choice}

To choose a random word from the histogram, the simplest algorithm
is to build a list with multiple copies of each word, according
to the observed frequency, and then choose from the list:

\begin{verbatim}
def random_word(h):
    t = []
    for word, freq in h.items():
        t.extend([word] * freq)

    return random.choice(t)
\end{verbatim}
%
The expression {\tt [word] * freq} creates a list with {\tt freq}
copies of the string {\tt word}.  The {\tt extend}
method is similar to {\tt append} except that the argument is
a sequence.

\begin{exercise}
\label{randhist}
\index{algorithm}

This algorithm works, but it is not very efficient; each time you
choose a random word, it rebuilds the list, which is as big as
the original book.  An obvious improvement is to build the list
once and then make multiple selections, but the list is still big.

An alternative is:

\begin{enumerate}

\item Use {\tt keys} to get a list of the words in the book.

\item Build a list that contains the cumulative sum of the word
  frequencies (see Exercise~\ref{cumulative}).  The last item
  in this list is the total number of words in the book, $n$.

\item Choose a random number from 1 to $n$.  Use a bisection search
  (See Exercise~\ref{bisection}) to find the index where the random
  number would be inserted in the cumulative sum.

\item Use the index to find the corresponding word in the word list.

\end{enumerate}

Write a program that uses this algorithm to choose a random
word from the book.  Solution: \url{http://thinkpython.com/code/analyze_book3.py}.

\end{exercise}



\section{Markov analysis}
\label{markov}
\index{Markov analysis}

If you choose words from the book at random, you can get a
sense of the vocabulary, you probably won't get a sentence:

\begin{verbatim}
this the small regard harriet which knightley's it most things
\end{verbatim}
%
A series of random words seldom makes sense because there
is no relationship between successive words.  For example, in
a real sentence you would expect an article like ``the'' to
be followed by an adjective or a noun, and probably not a verb
or adverb.

One way to measure these kinds of relationships is Markov
analysis, which
characterizes, for a given sequence of words, the probability of the
word that comes next.  For example, the song {\em Eric, the Half a
  Bee} begins:

\begin{quote}
Half a bee, philosophically, \\
Must, ipso facto, half not be. \\
But half the bee has got to be \\
Vis a vis, its entity. D'you see? \\
\\
But can a bee be said to be \\
Or not to be an entire bee \\
When half the bee is not a bee \\
Due to some ancient injury? \\
\end{quote}
%
In this text,
the phrase ``half the'' is always followed by the word ``bee,''
but the phrase ``the bee'' might be followed by either
``has'' or ``is''.
\index{prefix}
\index{suffix}
\index{mapping}

The result of Markov analysis is a mapping from each prefix
(like ``half the'' and ``the bee'') to all possible suffixes
(like ``has'' and ``is'').
\index{random text}
\index{text!random}

Given this mapping, you can generate a random text by
starting with any prefix and choosing at random from the
possible suffixes.  Next, you can combine the end of the
prefix and the new suffix to form the next prefix, and repeat.

For example, if you start with the prefix ``Half a,'' then the
next word has to be ``bee,'' because the prefix only appears
once in the text.  The next prefix is ``a bee,'' so the
next suffix might be ``philosophically,'' ``be'' or ``due.''

In this example the length of the prefix is always two, but
you can do Markov analysis with any prefix length.  The length
of the prefix is called the ``order'' of the analysis.

\begin{exercise}

Markov analysis:

\begin{enumerate}

\item Write a program to read a text from a file and perform Markov
analysis.  The result should be a dictionary that maps from
prefixes to a collection of possible suffixes.  The collection
might be a list, tuple, or dictionary; it is up to you to make
an appropriate choice.  You can test your program with prefix
length two, but you should write the program in a way that makes
it easy to try other lengths.

\item Add a function to the previous program to generate random text
based on the Markov analysis.  Here is an example from {\em Emma}
with prefix length 2:

\begin{quote}
He was very clever, be it sweetness or be angry, ashamed or only
amused, at such a stroke. She had never thought of Hannah till you
were never meant for me?" "I cannot make speeches, Emma:" he soon cut
it all himself.
\end{quote}

For this example, I left the punctuation attached to the words.
The result is almost syntactically correct, but not quite.
Semantically, it almost makes sense, but not quite.

What happens if you increase the prefix length?  Does the random
text make more sense?
\index{mash-up}

\item Once your program is working, you might want to try a mash-up:
if you analyze text from two or more books, the random
text you generate will blend the vocabulary and phrases from
the sources in interesting ways.

\end{enumerate}

Credit: This case study is based on an example from Kernighan and
Pike, {\em The Practice of Programming}, Addison-Wesley, 1999.

\end{exercise}

You should attempt this exercise before you go on; then you can can
download my solution from \url{http://thinkpython.com/code/markov.py}.  You
will also need \url{http://thinkpython.com/code/emma.txt}.


\section{Data structures}
\index{data structure}

Using Markov analysis to generate random text is fun, but there is
also a point to this exercise: data structure selection.  In your
solution to the previous exercises, you had to choose:

\begin{itemize}

\item How to represent the prefixes.

\item How to represent the collection of possible suffixes.

\item How to represent the mapping from each prefix to
the collection of possible suffixes.

\end{itemize}

Ok, the last one is easy; the only mapping type we have
seen is a dictionary, so it is the natural choice.

For the prefixes, the most obvious options are string,
list of strings, or tuple of strings.  For the suffixes,
one option is a list; another is a histogram (dictionary).
\index{implementation}

How should you choose?  The first step is to think about
the operations you will need to implement for each data structure.
For the prefixes, we need to be able to remove words from
the beginning and add to the end.  For example, if the current
prefix is ``Half a,'' and the next word is ``bee,'' you need
to be able to form the next prefix, ``a bee.''
\index{tuple!as key in dictionary}

Your first choice might be a list, since it is easy to add
and remove elements, but we also need to be able to use the
prefixes as keys in a dictionary, so that rules out lists.
With tuples, you can't append or remove, but you can use
the addition operator to form a new tuple:

\begin{verbatim}
def shift(prefix, word):
    return prefix[1:] + (word,)
\end{verbatim}
%
{\tt shift} takes a tuple of words, {\tt prefix}, and a string,
{\tt word}, and forms a new tuple that has all the words
in {\tt prefix} except the first, and {\tt word} added to
the end.

For the collection of suffixes, the operations we need to
perform include adding a new suffix (or increasing the frequency
of an existing one), and choosing a random suffix.

Adding a new suffix is equally easy for the list implementation
or the histogram.  Choosing a random element from a list
is easy; choosing from a histogram is harder to do
efficiently (see Exercise~\ref{randhist}).

So far we have been talking mostly about ease of implementation,
but there are other factors to consider in choosing data structures.
One is run time.  Sometimes there is a theoretical reason to expect
one data structure to be faster than other; for example, I mentioned
that the {\tt in} operator is faster for dictionaries than for lists,
at least when the number of elements is large.

But often you don't know ahead of time which implementation will
be faster.  One option is to implement both of them and see which
is better.  This approach is called {\bf benchmarking}.  A practical
alternative is to choose the data structure that is
easiest to implement, and then see if it is fast enough for the
intended application.  If so, there is no need to go on.  If not,
there are tools, like the {\tt profile} module, that can identify
the places in a program that take the most time.
\index{benchmarking}
\index{profile module}
\index{module!profile}

The other factor to consider is storage space.  For example, using a
histogram for the collection of suffixes might take less space because
you only have to store each word once, no matter how many times it
appears in the text.  In some cases, saving space can also make your
program run faster, and in the extreme, your program might not run at
all if you run out of memory.  But for many applications, space is a
secondary consideration after run time.

One final thought: in this discussion, I have implied that
we should use one data structure for both analysis and generation.  But
since these are separate phases, it would also be possible to use one
structure for analysis and then convert to another structure for
generation.  This would be a net win if the time saved during
generation exceeded the time spent in conversion.


\section{Debugging}
\index{debugging}

When you are debugging a program, and especially if you are
working on a hard bug, there are four things to try:

\begin{description}

\item[reading:] Examine your code, read it back to yourself, and
check that it says what you meant to say.

\item[running:] Experiment by making changes and running different
versions.  Often if you display the right thing at the right place
in the program, the problem becomes obvious, but sometimes you have to
spend some time to build scaffolding.

\item[ruminating:] Take some time to think!  What kind of error
is it: syntax, runtime, semantic?  What information can you get from
the error messages, or from the output of the program?  What kind of
error could cause the problem you're seeing?  What did you change
last, before the problem appeared?

\item[retreating:] At some point, the best thing to do is back
off, undoing recent changes, until you get back to a program that
works and that you understand.  Then you can start rebuilding.

\end{description}

Beginning programmers sometimes get stuck on one of these activities
and forget the others.  Each activity comes with its own failure
mode.
\index{typographical error}

For example, reading your code might help if the problem is a
typographical error, but not if the problem is a conceptual
misunderstanding.  If you don't understand what your program does, you
can read it 100 times and never see the error, because the error is in
your head.
\index{experimental debugging}

Running experiments can help, especially if you run small, simple
tests.  But if you run experiments without thinking or reading your
code, you might fall into a pattern I call ``random walk programming,''
which is the process of making random changes until the program
does the right thing.  Needless to say, random walk programming
can take a long time.
\index{random walk programming}
\index{development plan!random walk programming}

You have to take time to think.  Debugging is like an
experimental science.  You should have at least one hypothesis about
what the problem is.  If there are two or more possibilities, try to
think of a test that would eliminate one of them.

Taking a break helps with the thinking.  So does talking.
If you explain the problem to someone else (or even yourself), you
will sometimes find the answer before you finish asking the question.

But even the best debugging techniques will fail if there are too many
errors, or if the code you are trying to fix is too big and
complicated.  Sometimes the best option is to retreat, simplifying the
program until you get to something that works and that you
understand.

Beginning programmers are often reluctant to retreat because
they can't stand to delete a line of code (even if it's wrong).
If it makes you feel better, copy your program into another file
before you start stripping it down.  Then you can paste the pieces
back in a little bit at a time.

Finding a hard bug requires reading, running, ruminating, and
sometimes retreating.  If you get stuck on one of these activities,
try the others.


\section{Glossary}

\begin{description}

\item[deterministic:] Pertaining to a program that does the same
thing each time it runs, given the same inputs.
\index{deterministic}

\item[pseudorandom:] Pertaining to a sequence of numbers that appear
to be random, but are generated by a deterministic program.
\index{pseudorandom}

\item[default value:] The value given to an optional parameter if no
argument is provided.
\index{default value}

\item[override:] To replace a default value with an argument.
\index{override}

\item[benchmarking:] The process of choosing between data structures
by implementing alternatives and testing them on a sample of the
possible inputs.
\index{benchmarking}

\end{description}


\section{Exercises}

\begin{exercise}
\index{word frequency}
\index{frequency!word}
\index{Zipf's law}

The ``rank'' of a word is its position in a list of words
sorted by frequency: the most common word has rank 1, the
second most common has rank 2, etc.

Zipf's law describes a relationship between the ranks and frequencies
of words in natural languages
(\url{http://en.wikipedia.org/wiki/Zipf's_law}).  Specifically, it
predicts that the frequency, $f$, of the word with rank $r$ is:

\[ f = c r^{-s} \]
%
where $s$ and $c$ are parameters that depend on the language and the
text.  If you take the logarithm of both sides of this equation, you
get:
\index{logarithm}

\[ \log f = \log c - s \log r \]
%
So if you plot log $f$ versus log $r$, you should get
a straight line with slope $-s$ and intercept log $c$.

Write a program that reads a text from a file, counts
word frequencies, and prints one line
for each word, in descending order of frequency, with
log $f$ and log $r$.  Use the graphing program of your
choice to plot the results and check whether they form
a straight line.  Can you estimate the value of $s$?

Solution: \url{http://thinkpython.com/code/zipf.py}.  To make the plots, you
might have to install matplotlib (see
\url{http://matplotlib.sourceforge.net/}).
\index{Matplotlib}

\end{exercise}


\chapter{Classes and methods}

Code examples from this chapter are available from
\url{http://thinkpython.com/code/Time2.py}.

\section{Object-oriented features}
\index{object-oriented programming}

Python is an {\bf object-oriented programming language}, which means
that it provides features that support object-oriented
programming.

It is not easy to define object-oriented programming, but we have
already seen some of its characteristics:

\begin{itemize}

\item Programs are made up of object definitions and function
definitions, and most of the computation is expressed in terms
of operations on objects.

\item Each object definition corresponds to some object or concept
in the real world, and the functions that operate on that object
correspond to the ways real-world objects interact.

\end{itemize}

For example, the {\tt Time} class defined in Chapter~\ref{time}
corresponds to the way people record the time of day, and the
functions we defined correspond to the kinds of things people do with
times.  Similarly, the {\tt Point} and {\tt Rectangle} classes
correspond to the mathematical concepts of a point and a rectangle.

So far, we have not taken advantage of the features Python provides to
support object-oriented programming.  These
features are not strictly necessary; most of them provide
alternative syntax for things we have already done.  But in many cases,
the alternative is more concise and more accurately conveys the
structure of the program.

For example, in the {\tt Time} program, there is no obvious
connection between the class definition and the function definitions
that follow.  With some examination, it is apparent that every function
takes at least one {\tt Time} object as an argument.
\index{method}
\index{function}

This observation is the motivation for {\bf methods}; a method is
a function that is associated with a particular class.
We have seen methods for strings, lists, dictionaries and tuples.
In this chapter, we will define methods for user-defined types.
\index{syntax}
\index{semantics}

Methods are semantically the same as functions, but there are
two syntactic differences:

\begin{itemize}

\item Methods are defined inside a class definition in order
to make the relationship between the class and the method explicit.

\item The syntax for invoking a method is different from the
syntax for calling a function.

\end{itemize}

In the next few sections, we will take the functions from the previous
two chapters and transform them into methods.  This transformation is
purely mechanical; you can do it simply by following a sequence of
steps.  If you are comfortable converting from one form to another,
you will be able to choose the best form for whatever you are doing.


\section{Printing objects}
\index{object!printing}

In Chapter~\ref{time}, we defined a class named
{\tt Time} and in Exercise~\ref{ex.printtime}, you
wrote a function named \verb"print_time":

\begin{verbatim}
class Time(object):
    """Represents the time of day."""

def print_time(time):
    print '%.2d:%.2d:%.2d' % (time.hour, time.minute, time.second)
\end{verbatim}
%
To call this function, you have to pass a {\tt Time} object as an
argument:

\begin{verbatim}
>>> start = Time()
>>> start.hour = 9
>>> start.minute = 45
>>> start.second = 00
>>> print_time(start)
09:45:00
\end{verbatim}
%
To make \verb"print_time" a method, all we have to do is
move the function definition inside the class definition.  Notice
the change in indentation.
\index{indentation}

\begin{verbatim}
class Time(object):
    def print_time(time):
        print '%.2d:%.2d:%.2d' % (time.hour, time.minute, time.second)
\end{verbatim}
%
Now there are two ways to call \verb"print_time".  The first
(and less common) way is to use function syntax:
\index{function syntax}
\index{dot notation}


\begin{verbatim}
>>> Time.print_time(start)
09:45:00
\end{verbatim}
%
In this use of dot notation, {\tt Time} is the name of the class,
and \verb"print_time" is the name of the method.  {\tt start} is
passed as a parameter.

The second (and more concise) way is to use method syntax:
\index{method syntax}

\begin{verbatim}
>>> start.print_time()
09:45:00
\end{verbatim}
%
In this use of dot notation, \verb"print_time" is the name of the
method (again), and {\tt start} is the object the method is
invoked on, which is called the {\bf subject}.  Just as the
subject of a sentence is what the sentence is about, the subject
of a method invocation is what the method is about.
\index{subject}

Inside the method, the subject is assigned to the first
parameter, so in this case {\tt start} is assigned
to {\tt time}.
\index{self (parameter name)}
\index{parameter!self}

By convention, the first parameter of a method is
called {\tt self}, so it would be more common to write
\verb"print_time" like this:

\begin{verbatim}
class Time(object):
    def print_time(self):
        print '%.2d:%.2d:%.2d' % (self.hour, self.minute, self.second)
\end{verbatim}
%
The reason for this convention is an implicit metaphor:
\index{metaphor, method invocation}

\begin{itemize}

\item The syntax for a function call, \verb"print_time(start)",
  suggests that the function is the active agent.  It says something
  like, ``Hey \verb"print_time"!  Here's an object for you to print.''

\item In object-oriented programming, the objects are the active
  agents.  A method invocation like \verb"start.print_time()" says
  ``Hey {\tt start}!  Please print yourself.''

\end{itemize}

This change in perspective might be more polite, but it is not obvious
that it is useful.  In the examples we have seen so far, it may not
be.  But sometimes shifting responsibility from the functions onto the
objects makes it possible to write more versatile functions, and makes
it easier to maintain and reuse code.

\begin{exercise}
\label{convert}

Rewrite \verb"time_to_int" (from Section~\ref{prototype}) as a method.
It is probably not appropriate to rewrite \verb"int_to_time" as a
method; what object you would invoke it on?

\end{exercise}


\section{Another example}
\index{increment}

Here's a version of {\tt increment} (from Section~\ref{increment})
rewritten as a method:

\begin{verbatim}
# inside class Time:

    def increment(self, seconds):
        seconds += self.time_to_int()
        return int_to_time(seconds)
\end{verbatim}
%
This version assumes that \verb"time_to_int" is written
as a method, as in Exercise~\ref{convert}.  Also, note that
it is a pure function, not a modifier.

Here's how you would invoke {\tt increment}:

\begin{verbatim}
>>> start.print_time()
09:45:00
>>> end = start.increment(1337)
>>> end.print_time()
10:07:17
\end{verbatim}
%
The subject, {\tt start}, gets assigned to the first parameter,
{\tt self}.  The argument, {\tt 1337}, gets assigned to the
second parameter, {\tt seconds}.

This mechanism can be confusing, especially if you make an error.
For example, if you invoke {\tt increment} with two arguments, you
get:
\index{exception!TypeError}
\index{TypeError}

\begin{verbatim}
>>> end = start.increment(1337, 460)
TypeError: increment() takes exactly 2 arguments (3 given)
\end{verbatim}
%
The error message is initially confusing, because there are
only two arguments in parentheses.  But the subject is also
considered an argument, so all together that's three.


\section{A more complicated example}

\verb"is_after" (from Exercise~\ref{isafter}) is slightly more complicated
because it takes two Time objects as parameters.  In this case it is
conventional to name the first parameter {\tt self} and the second
parameter {\tt other}:
\index{other (parameter name)}
\index{parameter!other}

\begin{verbatim}
# inside class Time:

    def is_after(self, other):
        return self.time_to_int() > other.time_to_int()
\end{verbatim}
%
To use this method, you have to invoke it on one object and pass
the other as an argument:

\begin{verbatim}
>>> end.is_after(start)
True
\end{verbatim}
%
One nice thing about this syntax is that it almost reads
like English: ``end is after start?''


\section{The init method}
\index{init method}
\index{method!init}

The init method (short for ``initialization'') is
a special method that gets invoked when an object is instantiated.
Its full name is \verb"__init__" (two underscore characters,
followed by {\tt init}, and then two more underscores).  An
init method for the {\tt Time} class might look like this:

\begin{verbatim}
# inside class Time:

    def __init__(self, hour=0, minute=0, second=0):
        self.hour = hour
        self.minute = minute
        self.second = second
\end{verbatim}
%
It is common for the parameters of \verb"__init__"
to have the same names as the attributes.  The statement

\begin{verbatim}
        self.hour = hour
\end{verbatim}
%
stores the value of the parameter {\tt hour} as an attribute
of {\tt self}.
\index{optional parameter}
\index{parameter!optional}
\index{default value}
\index{override}

The parameters are optional, so if you call {\tt Time} with
no arguments, you get the default values.

\begin{verbatim}
>>> time = Time()
>>> time.print_time()
00:00:00
\end{verbatim}
%
If you provide one argument, it overrides {\tt hour}:

\begin{verbatim}
>>> time = Time (9)
>>> time.print_time()
09:00:00
\end{verbatim}
%
If you provide two arguments, they override {\tt hour} and
{\tt minute}.

\begin{verbatim}
>>> time = Time(9, 45)
>>> time.print_time()
09:45:00
\end{verbatim}
%
And if you provide three arguments, they override all three
default values.


\begin{exercise}
\index{Point class}
\index{class!Point}

Write an init method for the {\tt Point} class that takes
{\tt x} and {\tt y} as optional parameters and assigns
them to the corresponding attributes.
\end{exercise}


\section{The {\tt \_\_str\_\_} method}
\index{str method@\_\_str\_\_ method}
\index{method!\_\_str\_\_}

\verb"__str__" is a special method, like \verb"__init__",
that is supposed to return a string representation of an object.
\index{string representation}

For example, here is a {\tt str} method for Time objects:

\begin{verbatim}
# inside class Time:

    def __str__(self):
        return '%.2d:%.2d:%.2d' % (self.hour, self.minute, self.second)
\end{verbatim}
%
When you {\tt print} an object, Python invokes the {\tt str} method:
\index{print statement}
\index{statement!print}

\begin{verbatim}
>>> time = Time(9, 45)
>>> print time
09:45:00
\end{verbatim}
%
When I write a new class, I almost always start by writing
\verb"__init__", which makes it easier to instantiate objects, and
\verb"__str__", which is useful for debugging.


\begin{exercise}

Write a {\tt str} method for the {\tt Point} class.  Create
a Point object and print it.

\end{exercise}


\section{Operator overloading}
\label{operator.overloading}

By defining other special methods, you can specify the behavior
of operators on user-defined types.  For example, if you define
a method named \verb"__add__" for the {\tt Time} class, you can use the
{\tt +} operator on Time objects.

Here is what the definition might look like:
\index{add method}
\index{method!add}

\begin{verbatim}
# inside class Time:

    def __add__(self, other):
        seconds = self.time_to_int() + other.time_to_int()
        return int_to_time(seconds)
\end{verbatim}
%
And here is how you could use it:

\begin{verbatim}
>>> start = Time(9, 45)
>>> duration = Time(1, 35)
>>> print start + duration
11:20:00
\end{verbatim}
%
When you apply the {\tt +} operator to Time objects, Python invokes
\verb"__add__".  When you print the result, Python invokes
\verb"__str__".  So there is quite a lot happening behind the scenes!
\index{operator overloading}

Changing the behavior of an operator so that it works with
user-defined types is called {\bf operator overloading}.  For every
operator in Python there is a corresponding special method, like
\verb"__add__".  For more details, see
\url{http://docs.python.org/2/reference/datamodel.html#specialnames}.

\begin{exercise}

Write an {\tt add} method for the Point class.

\end{exercise}


\section{Type-based dispatch}

In the previous section we added two Time objects, but you
also might want to add an integer to a Time object.  The
following is a version of \verb"__add__"
that checks the type of {\tt other} and invokes either
\verb"add_time" or {\tt increment}:

\begin{verbatim}
# inside class Time:

    def __add__(self, other):
        if isinstance(other, Time):
            return self.add_time(other)
        else:
            return self.increment(other)

    def add_time(self, other):
        seconds = self.time_to_int() + other.time_to_int()
        return int_to_time(seconds)

    def increment(self, seconds):
        seconds += self.time_to_int()
        return int_to_time(seconds)
\end{verbatim}
%
The built-in function {\tt isinstance} takes a value and a
class object, and returns {\tt True} if the value is an instance
of the class.
\index{isinstance function}
\index{function!isinstance}

If {\tt other} is a Time object, \verb"__add__" invokes
\verb"add_time".  Otherwise it assumes that the parameter
is a number and invokes {\tt increment}.  This operation is
called a {\bf type-based dispatch} because it dispatches the
computation to different methods based on the type of the
arguments.
\index{type-based dispatch}
\index{dispatch, type-based}

Here are examples that use the {\tt +} operator with different
types:

\begin{verbatim}
>>> start = Time(9, 45)
>>> duration = Time(1, 35)
>>> print start + duration
11:20:00
>>> print start + 1337
10:07:17
\end{verbatim}
%
Unfortunately, this implementation of addition is not commutative.
If the integer is the first operand, you get
\index{commutativity}

\begin{verbatim}
>>> print 1337 + start
TypeError: unsupported operand type(s) for +: 'int' and 'instance'
\end{verbatim}
%
The problem is, instead of asking the Time object to add an integer,
Python is asking an integer to add a Time object, and it doesn't know
how to do that.  But there is a clever solution for this problem: the
special method \verb"__radd__", which stands for ``right-side add.''
This method is invoked when a Time object appears on the right side of
the {\tt +} operator.  Here's the definition:
\index{radd method}
\index{method!radd}

\begin{verbatim}
# inside class Time:

    def __radd__(self, other):
        return self.__add__(other)
\end{verbatim}
%
And here's how it's used:

\begin{verbatim}
>>> print 1337 + start
10:07:17
\end{verbatim}
%

\begin{exercise}

Write an {\tt add} method for Points that works with either a
Point object or a tuple:

\begin{itemize}

\item If the second operand is a Point, the method should return a new
Point whose $x$ coordinate is the sum of the $x$ coordinates of the
operands, and likewise for the $y$ coordinates.

\item If the second operand is a tuple, the method should add the
first element of the tuple to the $x$ coordinate and the second
element to the $y$ coordinate, and return a new Point with the result.

\end{itemize}

\end{exercise}

\section{Polymorphism}

Type-based dispatch is useful when it is necessary, but (fortunately)
it is not always necessary.  Often you can avoid it by writing functions
that work correctly for arguments with different types.
\index{type-based dispatch}
\index{dispatch!type-based}

Many of the functions we wrote for strings will actually
work for any kind of sequence.
For example, in Section~\ref{histogram}
we used {\tt histogram} to count the number of times each letter
appears in a word.

\begin{verbatim}
def histogram(s):
    d = dict()
    for c in s:
        if c not in d:
            d[c] = 1
        else:
            d[c] = d[c]+1
    return d
\end{verbatim}
%
This function also works for lists, tuples, and even dictionaries,
as long as the elements of {\tt s} are hashable, so they can be used
as keys in {\tt d}.

\begin{verbatim}
>>> t = ['spam', 'egg', 'spam', 'spam', 'bacon', 'spam']
>>> histogram(t)
{'bacon': 1, 'egg': 1, 'spam': 4}
\end{verbatim}
%
Functions that can work with several types are called {\bf polymorphic}.
Polymorphism can facilitate code reuse.  For example, the built-in
function {\tt sum}, which adds the elements of a sequence, works
as long as the elements of the sequence support addition.
\index{polymorphism}

Since Time objects provide an {\tt add} method, they work
with {\tt sum}:

\begin{verbatim}
>>> t1 = Time(7, 43)
>>> t2 = Time(7, 41)
>>> t3 = Time(7, 37)
>>> total = sum([t1, t2, t3])
>>> print total
23:01:00
\end{verbatim}
%
In general, if all of the operations inside a function
work with a given type, then the function works with that type.

The best kind of polymorphism is the unintentional kind, where
you discover that a function you already wrote can be
applied to a type you never planned for.


\section{Debugging}
\index{debugging}

It is legal to add attributes to objects at any point in the execution
of a program, but if you are a stickler for type theory, it is a
dubious practice to have objects of the same type with different
attribute sets.  It is usually a good idea to
initialize all of an object's attributes in the init method.
\index{init method}
\index{attribute!initializing}

If you are not sure whether an object has a particular attribute, you
can use the built-in function {\tt hasattr} (see Section~\ref{hasattr}).
\index{hasattr function}
\index{function!hasattr}
\index{dict attribute@\_\_dict\_\_ attribute}
\index{attribute!\_\_dict\_\_}

Another way to access the attributes of an object is through the
special attribute \verb"__dict__", which is a dictionary that maps
attribute names (as strings) and values:

\begin{verbatim}
>>> p = Point(3, 4)
>>> print p.__dict__
{'y': 4, 'x': 3}
\end{verbatim}
%
For purposes of debugging, you might find it useful to keep this
function handy:

\begin{verbatim}
def print_attributes(obj):
    for attr in obj.__dict__:
        print attr, getattr(obj, attr)
\end{verbatim}
%
\verb"print_attributes" traverses the items in the object's dictionary
and prints each attribute name and its corresponding value.
\index{traversal!dictionary}
\index{dictionary!traversal}

The built-in function {\tt getattr} takes an object and an attribute
name (as a string) and returns the attribute's value.
\index{getattr function}
\index{function!getattr}


\section{Interface and implementation}

One of the goals of object-oriented design is to make software more
maintainable, which means that you can keep the program working when
other parts of the system change, and modify the program to meet new
requirements.
\index{interface}
\index{implementation}
\index{maintainable}
\index{object-oriented design}

A design principle that helps achieve that goal is to keep
interfaces separate from implementations.  For objects, that means
that the methods a class provides should not depend on how the
attributes are represented.
\index{attribute}

For example, in this chapter we developed a class that represents
a time of day.  Methods provided by this class include
\verb"time_to_int", \verb"is_after", and \verb"add_time".

We could implement those methods in several ways.  The details of the
implementation depend on how we represent time.  In this chapter, the
attributes of a {\tt Time} object are {\tt hour}, {\tt minute}, and
{\tt second}.

As an alternative, we could replace these attributes with
a single integer representing the number of seconds
since midnight.  This implementation would make some methods,
like \verb"is_after", easier to write, but it makes some methods
harder.

After you deploy a new class, you might discover a better
implementation.  If other parts of the program are using your
class, it might be time-consuming and error-prone to change the
interface.

But if you designed the interface carefully, you can
change the implementation without changing the interface, which
means that other parts of the program don't have to change.

Keeping the interface separate from the implementation means that
you have to hide the attributes.  Code in other parts of the program
(outside the class definition) should use methods to read
and modify the state of the object.  They should not access the
attributes directly.  This principle is called {\bf information hiding};
see \url{http://en.wikipedia.org/wiki/Information_hiding}.
\index{information hiding}

\begin{exercise}

Download the code from this chapter
(\url{http://thinkpython.com/code/Time2.py}).  Change the attributes
of {\tt Time} to be a single integer representing seconds since
midnight.  Then modify the methods (and the function
\verb"int_to_time") to work with the new implementation.  You should
not have to modify the test code in {\tt main}.  When you are done,
the output should be the same as before.  Solution:
\url{http://thinkpython.com/code/Time2_soln.py}

\end{exercise}


\section{Glossary}

\begin{description}

\item[object-oriented language:] A language that provides features,
  such as user-defined classes and method syntax, that facilitate
  object-oriented programming.
\index{object-oriented language}

\item[object-oriented programming:] A style of programming in which
data and the operations that manipulate it are organized into classes
and methods.
\index{object-oriented programming}

\item[method:] A function that is defined inside a class definition and
is invoked on instances of that class.
\index{method}

\item[subject:] The object a method is invoked on.
\index{subject}

\item[operator overloading:] Changing the behavior of an operator like
{\tt +} so it works with a user-defined type.
\index{overloading}
\index{operator!overloading}

\item[type-based dispatch:] A programming pattern that checks the type
of an operand and invokes different functions for different types.
\index{type-based dispatch}

\item[polymorphic:] Pertaining to a function that can work with more
  than one type.
\index{polymorphism}

\item[information hiding:] The principle that the interface provided
by an object should not depend on its implementation, in particular
the representation of its attributes.
\index{information hiding}


\end{description}

\section{Exercises}

\begin{exercise}
\index{default value!avoiding mutable}
\index{mutable object, as default value}
\index{worst bug}
\index{bug!worst}
\index{Kangaroo class}
\index{class!Kangaroo}

This exercise is a cautionary tale about one of the most
common, and difficult to find, errors in Python.
Write a definition for a class named {\tt Kangaroo} with the following
methods:

\begin{enumerate}

\item An \verb"__init__" method that initializes an attribute named
\verb"pouch_contents" to an empty list.

\item A method named \verb"put_in_pouch" that takes an object
of any type and adds it to \verb"pouch_contents".

\item A \verb"__str__" method that returns a string representation
of the Kangaroo object and the contents of the pouch.

\end{enumerate}
%
Test your code
by creating two {\tt Kangaroo} objects, assigning them to variables
named {\tt kanga} and {\tt roo}, and then adding {\tt roo} to the
contents of {\tt kanga}'s pouch.

Download \url{http://thinkpython.com/code/BadKangaroo.py}.  It contains
a solution to the previous problem with one big, nasty bug.
Find and fix the bug.

If you get stuck, you can download
\url{http://thinkpython.com/code/GoodKangaroo.py}, which explains the
problem and demonstrates a solution.
\index{aliasing}
\index{embedded object}
\index{object!embedded}

\end{exercise}




\begin{exercise}
\index{Visual module}
\index{module!Visual}
\index{vpython module}
\index{module!vpython}

Visual is a Python module that provides 3-D graphics.  It is
not always included in a Python installation, so you might have
to install it from your software repository or, if it's not there,
from \url{http://vpython.org}.

The following example creates a 3-D space that is 256 units
wide, long and high, and sets the ``center'' to be the
point $(128,128,128)$.  Then it draws a blue sphere.

\begin{verbatim}
from visual import *

scene.range = (256, 256, 256)
scene.center = (128, 128, 128)

color = (0.1, 0.1, 0.9)          # mostly blue
sphere(pos=scene.center, radius=128, color=color)
\end{verbatim}

{\tt color} is an RGB tuple; that is, the elements are Red-Green-Blue
levels between 0.0 and 1.0 (see
\url{http://en.wikipedia.org/wiki/RGB_color_model}).

If you run this code, you should see a window with a black
background and a blue sphere.  If you drag the middle button
up and down, you can zoom in and out.  You can also rotate
the scene by dragging the right button, but with only one
sphere in the world, it is hard to tell the difference.

The following loop creates a cube of spheres:

\begin{verbatim}
t = range(0, 256, 51)
for x in t:
    for y in t:
        for z in t:
            pos = x, y, z
            sphere(pos=pos, radius=10, color=color)
\end{verbatim}

\begin{enumerate}

\item Put this code in a script and make sure it works for
you.

\item Modify the program so that each sphere in the cube
has the color that corresponds to its position in RGB space.
Notice that the coordinates are in the range 0--255, but
the RGB tuples are in the range 0.0--1.0.
\index{color list}
\index{available colors}

\item Download \url{http://thinkpython.com/code/color_list.py}
and use the function \verb"read_colors" to generate a list
of the available colors on your system, their names and
RGB values.  For each named color draw a sphere in the
position that corresponds to its RGB values.



\end{enumerate}

You can see my solution at \url{http://thinkpython.com/code/color_space.py}.

\end{exercise}


\chapter{Inheritance}

In this chapter I present classes to represent playing cards,
decks of cards, and poker hands.  If you don't play poker, you can
read about it at \url{http://en.wikipedia.org/wiki/Poker}, but you don't have
to; I'll tell you what you need to know for the exercises.
Code examples from this chapter are available from
\url{http://thinkpython.com/code/Card.py}.
\index{playing card, Anglo-American}
\index{card, playing}
\index{poker}

If you are not familiar with Anglo-American playing cards,
you can read about them at \url{http://en.wikipedia.org/wiki/Playing_cards}.


\section{Card objects}

There are fifty-two cards in a deck, each of which belongs to one of
four suits and one of thirteen ranks.  The suits are Spades, Hearts,
Diamonds, and Clubs (in descending order in bridge).  The ranks are
Ace, 2, 3, 4, 5, 6, 7, 8, 9, 10, Jack, Queen, and King.  Depending on
the game that you are playing, an Ace may be higher than King
or lower than 2.
\index{rank}
\index{suit}

If we want to define a new object to represent a playing card, it is
obvious what the attributes should be: {\tt rank} and
{\tt suit}.  It is not as obvious what type the attributes
should be.  One possibility is to use strings containing words like
\verb"'Spade'" for suits and \verb"'Queen'" for ranks.  One problem with
this implementation is that it would not be easy to compare cards to
see which had a higher rank or suit.
\index{encode}
\index{encrypt}
\index{map to}
\index{representation}

An alternative is to use integers to {\bf encode} the ranks and suits.
In this context, ``encode'' means that we are going to define a mapping
between numbers and suits, or between numbers and ranks.  This
kind of encoding is not meant to be a secret (that
would be ``encryption'').

\newcommand{\mymapsto}{$\mapsto$}

For example, this table shows the suits and the corresponding integer
codes:

\begin{tabular}{l c l}
Spades & \mymapsto & 3 \\
Hearts & \mymapsto & 2 \\
Diamonds & \mymapsto & 1 \\
Clubs & \mymapsto & 0
\end{tabular}

This code makes it easy to compare cards; because higher suits map to
higher numbers, we can compare suits by comparing their codes.

The mapping for ranks is fairly obvious; each of the numerical ranks
maps to the corresponding integer, and for face cards:

\begin{tabular}{l c l}
Jack & \mymapsto & 11 \\
Queen & \mymapsto & 12 \\
King & \mymapsto & 13 \\
\end{tabular}

I am using the \mymapsto~symbol to make it clear that these mappings
are not part of the Python program.  They are part of the program
design, but they don't appear explicitly in the code.
\index{Card class}
\index{class!Card}

The class definition for {\tt Card} looks like this:

\begin{verbatim}
class Card(object):
    """Represents a standard playing card."""

    def __init__(self, suit=0, rank=2):
        self.suit = suit
        self.rank = rank
\end{verbatim}
%
As usual, the init method takes an optional
parameter for each attribute.  The default card is
the 2 of Clubs.
\index{init method}
\index{method!init}

To create a Card, you call {\tt Card} with the
suit and rank of the card you want.

\begin{verbatim}
queen_of_diamonds = Card(1, 12)
\end{verbatim}
%


\section{Class attributes}
\label{class.attribute}
\index{class attribute}
\index{attribute!class}

In order to print Card objects in a way that people can easily
read, we need a mapping from the integer codes to the corresponding
ranks and suits.  A natural way to
do that is with lists of strings.  We assign these lists to {\bf class
attributes}:

\begin{verbatim}
# inside class Card:

    suit_names = ['Clubs', 'Diamonds', 'Hearts', 'Spades']
    rank_names = [None, 'Ace', '2', '3', '4', '5', '6', '7',
              '8', '9', '10', 'Jack', 'Queen', 'King']

    def __str__(self):
        return '%s of %s' % (Card.rank_names[self.rank],
                             Card.suit_names[self.suit])
\end{verbatim}
%
Variables like \verb"suit_names" and \verb"rank_names", which are
defined inside a class but outside of any method, are called
class attributes because they are associated with the class object
{\tt Card}.
\index{instance attribute}
\index{attribute!instance}

This term distinguishes them from variables like {\tt suit} and {\tt
  rank}, which are called {\bf instance attributes} because they are
associated with a particular instance.
\index{dot notation}

Both kinds of attribute are accessed using dot notation.  For
example, in \verb"__str__", {\tt self} is a Card object,
and {\tt self.rank} is its rank.  Similarly, {\tt Card}
is a class object, and \verb"Card.rank_names" is a
list of strings associated with the class.

Every card has its own {\tt suit} and {\tt rank}, but there
is only one copy of \verb"suit_names" and \verb"rank_names".

Putting it all together, the expression
\verb"Card.rank_names[self.rank]" means ``use the attribute {\tt rank}
from the object {\tt self} as an index into the list \verb"rank_names"
from the class {\tt Card}, and select the appropriate string.''

The first element of \verb"rank_names" is {\tt None} because there
is no card with rank zero.  By including {\tt None} as a place-keeper,
we get a mapping with the nice property that the index 2 maps to the
string \verb"'2'", and so on.  To avoid this tweak, we could have
used a dictionary instead of a list.

With the methods we have so far, we can create and print cards:

\begin{verbatim}
>>> card1 = Card(2, 11)
>>> print card1
Jack of Hearts
\end{verbatim}

\begin{figure}
\centerline
{\includegraphics[scale=0.8]{figs/card1.pdf}}
\caption{Object diagram.}
\label{fig.card1}
\end{figure}

Figure~\ref{fig.card1} is a diagram of the {\tt Card} class object
and one Card instance.
\index{state diagram}
\index{diagram!state}
\index{object diagram}
\index{diagram!object}
{\tt Card} is a class object, so it has type {\tt type}.  {\tt
card1} has type {\tt Card}.  (To save space, I didn't draw the
contents of \verb"suit_names" and \verb"rank_names").


\section{Comparing cards}
\label{comparecard}
\index{operator!relational}
\index{relational operator}

For built-in types, there are relational operators
({\tt <}, {\tt >}, {\tt ==}, etc.)
that compare
values and determine when one is greater than, less than, or equal to
another.  For user-defined types, we can override the behavior of
the built-in operators by providing a method named
\verb"__cmp__".

\verb"__cmp__" takes two parameters, {\tt self} and {\tt other},
and returns a positive number if the first object is greater, a
negative number if the second object is greater, and 0 if they are
equal to each other.
\index{override}
\index{operator overloading}

The correct ordering for cards is not obvious.
For example, which
is better, the 3 of Clubs or the 2 of Diamonds?  One has a higher
rank, but the other has a higher suit.  In order to compare
cards, you have to decide whether rank or suit is more important.

The answer might depend on what game you are playing, but to keep
things simple, we'll make the arbitrary choice that suit is more
important, so all of the Spades outrank all of the Diamonds,
and so on.
\index{cmp method@\_\_cmp\_\_ method}
\index{method!\_\_cmp\_\_}

With that decided, we can write \verb"__cmp__":

\begin{verbatim}
# inside class Card:

    def __cmp__(self, other):
        # check the suits
        if self.suit > other.suit: return 1
        if self.suit < other.suit: return -1

        # suits are the same... check ranks
        if self.rank > other.rank: return 1
        if self.rank < other.rank: return -1

        # ranks are the same... it's a tie
        return 0
\end{verbatim}
%
You can write this more concisely using tuple comparison:
\index{tuple!comparison}
\index{comparison!tuple}

\begin{verbatim}
# inside class Card:

    def __cmp__(self, other):
        t1 = self.suit, self.rank
        t2 = other.suit, other.rank
        return cmp(t1, t2)
\end{verbatim}
%
The built-in function {\tt cmp} has the same interface as
the method \verb"__cmp__": it takes two values and returns
a positive number if the first is larger, a negative number
if the second is larger, and 0 if they are equal.
\index{cmp function}
\index{function!cmp}

In Python 3, {\tt cmp} no longer exists, and the \verb"__cmp__"
method is not supported.  Instead you should provide \verb"__lt__",
which returns {\tt True} if {\tt self} is less than {\tt other}.
You can implement \verb"__lt__" using tuples and the \verb"<"
operator.

\begin{exercise}

Write a \verb"__cmp__" method for Time objects.  Hint: you
can use tuple comparison, but you also might consider using
integer subtraction.

%    def __cmp__(self, other):
%        return time_to_int(self) - time_to_int(other)

%If {\tt self} is later than {\tt other}, the result is
%a positive number.  If {\tt other} is later, the result
%is negative.  And if {\tt self} and {\tt other} are equal
%(but not necessarily identical)
%the result is zero.

\end{exercise}


\section{Decks}
\index{list!of objects}
\index{deck, playing cards}

Now that we have Cards, the next step is to define Decks.  Since a
deck is made up of cards, it is natural for each Deck to contain a
list of cards as an attribute.
\index{init method}
\index{method!init}

The following is a class definition for {\tt Deck}.  The
init method creates the attribute {\tt cards} and generates
the standard set of fifty-two cards:
\index{composition}
\index{loop!nested}
\index{Deck class}
\index{class!Deck}

\begin{verbatim}
class Deck(object):

    def __init__(self):
        self.cards = []
        for suit in range(4):
            for rank in range(1, 14):
                card = Card(suit, rank)
                self.cards.append(card)
\end{verbatim}
%
The easiest way to populate the deck is with a nested loop.  The outer
loop enumerates the suits from 0 to 3.  The inner loop enumerates the
ranks from 1 to 13.  Each iteration
creates a new Card with the current suit and rank,
and appends it to {\tt self.cards}.
\index{append method}
\index{method!append}


\section{Printing the deck}
\label{printdeck}
\index{str method@\_\_str\_\_ method}
\index{method!\_\_str\_\_}

Here is a \verb"__str__" method for {\tt Deck}:

\begin{verbatim}
#inside class Deck:

    def __str__(self):
        res = []
        for card in self.cards:
            res.append(str(card))
        return '\n'.join(res)
\end{verbatim}
%
This method demonstrates an efficient way to accumulate a large
string: building a list of strings and then using {\tt join}.
The built-in function {\tt str} invokes the \verb"__str__"
method on each card and returns the string representation.
\index{accumulator!string}
\index{string!accumulator}
\index{join method}
\index{method!join}
\index{newline}

Since we invoke {\tt join} on a newline character, the cards
are separated by newlines.  Here's what the result looks like:

\begin{verbatim}
>>> deck = Deck()
>>> print deck
Ace of Clubs
2 of Clubs
3 of Clubs
...
10 of Spades
Jack of Spades
Queen of Spades
King of Spades
\end{verbatim}
%
Even though the result appears on 52 lines, it is
one long string that contains newlines.


\section{Add, remove, shuffle and sort}

To deal cards, we would like a method that
removes a card from the deck and returns it.
The list method {\tt pop} provides a convenient way to do that:
\index{pop method}
\index{method!pop}

\begin{verbatim}
#inside class Deck:

    def pop_card(self):
        return self.cards.pop()
\end{verbatim}
%
Since {\tt pop} removes the {\em last} card in the list, we are
dealing from the bottom of the deck.  In real life ``bottom dealing'' is
frowned upon,
but in this context it's ok.
\index{append method}
\index{method!append}

To add a card, we can use the list method {\tt append}:

\begin{verbatim}
#inside class Deck:

    def add_card(self, card):
        self.cards.append(card)
\end{verbatim}
%
A method like this that uses another function without doing
much real work is sometimes called a {\bf veneer}.  The metaphor
comes from woodworking, where it is common to glue a thin
layer of good quality wood to the surface of a cheaper piece of
wood.
\index{veneer}

In this case we are defining a ``thin'' method that expresses
a list operation in terms that are appropriate for decks.

As another example, we can write a Deck method named {\tt shuffle}
using the function {\tt shuffle} from the {\tt random} module:
\index{random module}
\index{module!random}
\index{shuffle function}
\index{function!shuffle}

\begin{verbatim}
# inside class Deck:

    def shuffle(self):
        random.shuffle(self.cards)
\end{verbatim}
%
Don't forget to import {\tt random}.

\begin{exercise}
\index{sort method}
\index{method!sort}

Write a Deck method named {\tt sort} that uses the list method
{\tt sort} to sort the cards in a {\tt Deck}.  {\tt sort} uses
the \verb"__cmp__" method we defined to determine sort order.
\end{exercise}



\section{Inheritance}
\index{inheritance}
\index{object-oriented programming}

The language feature most often associated with object-oriented
programming is {\bf inheritance}.  Inheritance is the ability to
define a new class that is a modified version of an existing
class.
\index{parent class}
\index{child class}
\index{class!child}
\index{subclass}
\index{superclass}

It is called ``inheritance'' because the new class inherits the
methods of the existing class.  Extending this metaphor, the existing
class is called the {\bf parent} and the new class is
called the {\bf child}.

As an example, let's say we want a class to represent a ``hand,''
that is, the set of cards held by one player.  A hand is similar to a
deck: both are made up of a set of cards, and both require operations
like adding and removing cards.

A hand is also different from a deck; there are operations we want for
hands that don't make sense for a deck.  For example, in poker we
might compare two hands to see which one wins.  In bridge, we might
compute a score for a hand in order to make a bid.

This relationship between classes---similar, but different---lends
itself to inheritance.

The definition of a child class is like other class definitions,
but the name of the parent class appears in parentheses:
\index{parentheses!parent class in}
\index{parent class}
\index{class!parent}
\index{Hand class}
\index{class!Hand}

\begin{verbatim}
class Hand(Deck):
    """Represents a hand of playing cards."""
\end{verbatim}
%
This definition indicates that {\tt Hand} inherits from {\tt Deck};
that means we can use methods like \verb"pop_card" and \verb"add_card"
for Hands as well as Decks.

{\tt Hand} also inherits \verb"__init__" from {\tt Deck}, but
it doesn't really do what we want: instead of populating the hand
with 52 new cards, the init method for Hands should initialize
{\tt cards} with an empty list.
\index{override}
\index{init method}
\index{method!init}

If we provide an init method in the {\tt Hand} class, it overrides the
one in the {\tt Deck} class:

\begin{verbatim}
# inside class Hand:

    def __init__(self, label=''):
        self.cards = []
        self.label = label
\end{verbatim}
%
So when you create a Hand, Python invokes this init method:

\begin{verbatim}
>>> hand = Hand('new hand')
>>> print hand.cards
[]
>>> print hand.label
new hand
\end{verbatim}
%
But the other methods are inherited from {\tt Deck}, so we can use
\verb"pop_card" and \verb"add_card" to deal a card:

\begin{verbatim}
>>> deck = Deck()
>>> card = deck.pop_card()
>>> hand.add_card(card)
>>> print hand
King of Spades
\end{verbatim}
%
A natural next step is to encapsulate this code in a method
called \verb"move_cards":
\index{encapsulation}

\begin{verbatim}
#inside class Deck:

    def move_cards(self, hand, num):
        for i in range(num):
            hand.add_card(self.pop_card())
\end{verbatim}
%
\verb"move_cards" takes two arguments, a Hand object and the number of
cards to deal.  It modifies both {\tt self} and {\tt hand}, and
returns {\tt None}.

In some games, cards are moved from one hand to another,
or from a hand back to the deck.  You can use \verb"move_cards"
for any of these operations: {\tt self} can be either a Deck
or a Hand, and {\tt hand}, despite the name, can also be a {\tt Deck}.

\begin{exercise}

Write a Deck method called \verb"deal_hands" that takes two
parameters, the number of hands and the number of cards per
hand, and that creates new Hand objects, deals the appropriate
number of cards per hand, and returns a list of Hand objects.

\end{exercise}

Inheritance is a useful feature.  Some programs that would be
repetitive without inheritance can be written more elegantly
with it.  Inheritance can facilitate code reuse, since you can
customize the behavior of parent classes without having to modify
them.  In some cases, the inheritance structure reflects the natural
structure of the problem, which makes the program easier to
understand.

On the other hand, inheritance can make programs difficult to read.
When a method is invoked, it is sometimes not clear where to find its
definition.  The relevant code may be scattered among several modules.
Also, many of the things that can be done using inheritance can be
done as well or better without it.


\section{Class diagrams}
\label{class.diagram}

So far we have seen stack diagrams, which show the state of
a program, and object diagrams, which show the attributes
of an object and their values.  These diagrams represent a snapshot
in the execution of a program, so they change as the program
runs.

They are also highly detailed; for some purposes, too
detailed.  A class diagram is a more abstract representation
of the structure of a program.  Instead of showing individual
objects, it shows classes and the relationships between them.

There are several kinds of relationship between classes:

\begin{itemize}

\item Objects in one class might contain references to objects
in another class.  For example, each Rectangle contains a reference
to a Point, and each Deck contains references to many Cards.
This kind of relationship is called {\bf HAS-A}, as in, ``a Rectangle
has a Point.''

\item One class might inherit from another.  This relationship
is called {\bf IS-A}, as in, ``a Hand is a kind of a Deck.''

\item One class might depend on another in the sense that changes
in one class would require changes in the other.

\end{itemize}
\index{IS-A relationship}
\index{HAS-A relationship}
\index{class diagram}
\index{diagram!class}

A {\bf class diagram} is a graphical representation of these
relationships.  For example, Figure~\ref{fig.class1} shows the
relationships between {\tt Card}, {\tt Deck} and {\tt Hand}.

\begin{figure}
\centerline
{\includegraphics[scale=0.8]{figs/class1.pdf}}
\caption{Class diagram.}
\label{fig.class1}
\end{figure}


The arrow with a hollow triangle head represents an IS-A
relationship; in this case it indicates that Hand inherits
from Deck.

The standard arrow head represents a HAS-A
relationship; in this case a Deck has references to Card
objects.
\index{multiplicity (in class diagram)}

The star ({\tt *}) near the arrow head is a
{\bf multiplicity}; it indicates how many Cards a Deck has.
A multiplicity can be a simple number, like {\tt 52}, a range,
like {\tt 5..7} or a star, which indicates that a Deck can
have any number of Cards.

A more detailed diagram might show that a Deck actually
contains a {\em list} of Cards, but built-in types
like list and dict are usually not included in class diagrams.

\begin{exercise}

Read {\tt TurtleWorld.py}, {\tt World.py} and {\tt Gui.py}
and draw a class diagram that shows the relationships among
the classes defined there.

\end{exercise}


\section{Debugging}
\index{debugging}

Inheritance can make debugging a challenge because when you
invoke a method on an object, you might not know which method
will be invoked.
\index{polymorphism}

Suppose you are writing a function that works with Hand objects.
You would like it to work with all kinds of Hands, like
PokerHands, BridgeHands, etc.  If you invoke a method like
{\tt shuffle}, you might get the one defined in {\tt Deck},
but if any of the subclasses override this method, you'll
get that version instead.
\index{flow of execution}

Any time you are unsure about the flow of execution through your
program, the simplest solution is to add print statements at the
beginning of the relevant methods.  If {\tt Deck.shuffle} prints a
message that says something like {\tt Running Deck.shuffle}, then as
the program runs it traces the flow of execution.

As an alternative, you could use this function, which takes an
object and a method name (as a string) and returns the class that
provides the definition of the method:

\begin{verbatim}
def find_defining_class(obj, meth_name):
    for ty in type(obj).mro():
        if meth_name in ty.__dict__:
            return ty
\end{verbatim}
%
Here's an example:

\begin{verbatim}
>>> hand = Hand()
>>> print find_defining_class(hand, 'shuffle')
<class 'Card.Deck'>
\end{verbatim}
%
So the {\tt shuffle} method for this Hand is the one in {\tt Deck}.
\index{mro method}
\index{method!mro}
\index{method resolution order}

\verb"find_defining_class" uses the {\tt mro} method to get the list
of class objects (types) that will be searched for methods.  ``MRO''
stands for ``method resolution order.''
\index{override}
\index{interface}
\index{precondition}
\index{postcondition}

Here's a program design suggestion: whenever you override a method,
the interface of the new method should be the same as the old.  It
should take the same parameters, return the same type, and obey the
same preconditions and postconditions.  If you obey this rule, you
will find that any function designed to work with an instance of a
superclass, like a Deck, will also work with instances of subclasses
like a Hand or PokerHand.

If you violate this rule, your code will collapse like (sorry)
a house of cards.


\section{Data encapsulation}

Chapter~\ref{time} demonstrates a development plan we might call
``object-oriented design.''  We identified objects we needed---{\tt
  Time}, {\tt Point} and {\tt Rectangle}---and defined classes to
represent them.  In each case there is an obvious correspondence
between the object and some entity in the real world (or at least a
mathematical world).
\index{development plan}

But sometimes it is less obvious what objects you need
and how they should interact.  In that case you need a different
development plan.  In the same way that we discovered function
interfaces by encapsulation and generalization, we can discover
class interfaces by {\bf data encapsulation}.
\index{data encapsulation}
\index{encapsulation!data}

Markov analysis, from Section~\ref{markov}, provides a good example.
If you download my code from \url{http://thinkpython.com/code/markov.py},
you'll see that it uses two global variables---\verb"suffix_map" and
\verb"prefix"---that are read and written from several functions.

\begin{verbatim}
suffix_map = {}
prefix = ()
\end{verbatim}

Because these variables are global
we can only run one analysis
at a time.  If we read two texts, their prefixes and suffixes would
be added to the same data structures (which makes for some interesting
generated text).

To run multiple analyses, and keep them separate, we can encapsulate
the state of each analysis in an object.
Here's what that looks like:

\begin{verbatim}
class Markov(object):

    def __init__(self):
        self.suffix_map = {}
        self.prefix = ()
\end{verbatim}

Next, we transform the functions into methods.  For example,
here's \verb"process_word":

\begin{verbatim}
    def process_word(self, word, order=2):
        if len(self.prefix) < order:
            self.prefix += (word,)
            return

        try:
            self.suffix_map[self.prefix].append(word)
        except KeyError:
            # if there is no entry for this prefix, make one
            self.suffix_map[self.prefix] = [word]

        self.prefix = shift(self.prefix, word)
\end{verbatim}

Transforming a program like this---changing the design without
changing the function---is another example of refactoring
(see Section~\ref{refactoring}).
\index{refactoring}

This example suggests a development plan for designing objects and
methods:

\begin{enumerate}

\item Start by writing functions that read and write global
variables (when necessary).

\item Once you get the program working, look for associations
between global variables and the functions that use them.

\item Encapsulate related variables as attributes of an object.

\item Transform the associated functions into methods of the new
class.

\end{enumerate}


\begin{exercise}

Download my code from Section~\ref{markov}
(\url{http://thinkpython.com/code/markov.py}), and follow the steps described
above to encapsulate the global variables as attributes of a new class
called {\tt Markov}.  Solution: \url{http://thinkpython.com/code/Markov.py}
(note the capital M).

\end{exercise}




\section{Glossary}

\begin{description}

\item[encode:]  To represent one set of values using another
set of values by constructing a mapping between them.
\index{encode}

\item[class attribute:] An attribute associated with a class
object.  Class attributes are defined inside
a class definition but outside any method.
\index{class attribute}
\index{attribute!class}

\item[instance attribute:] An attribute associated with an
instance of a class.
\index{instance attribute}
\index{attribute!instance}

\item[veneer:] A method or function that provides a different
interface to another function without doing much computation.
\index{veneer}

\item[inheritance:] The ability to define a new class that is a
modified version of a previously defined class.
\index{inheritance}

\item[parent class:] The class from which a child class inherits.
\index{parent class}

\item[child class:] A new class created by inheriting from an
existing class; also called a ``subclass.''
\index{child class}
\index{class!child}

\item[IS-A relationship:] The relationship between a child class
and its parent class.
\index{IS-A relationship}

\item[HAS-A relationship:] The relationship between two classes
where instances of one class contain references to instances of
the other.
\index{HAS-A relationship}

\item[class diagram:] A diagram that shows the classes in a program
and the relationships between them.
\index{class diagram}
\index{diagram!class}

\item[multiplicity:] A notation in a class diagram that shows, for
a HAS-A relationship, how many references there are to instances
of another class.
\index{multiplicity (in class diagram)}

\end{description}


\section{Exercises}

\begin{exercise}
\label{poker}

The following are the possible hands in poker, in increasing order
of value (and decreasing order of probability):
\index{poker}

\begin{description}

\item[pair:] two cards with the same rank
\vspace{-0.05in}

\item[two pair:] two pairs of cards with the same rank
\vspace{-0.05in}

\item[three of a kind:] three cards with the same rank
\vspace{-0.05in}

\item[straight:] five cards with ranks in sequence (aces can
be high or low, so {\tt Ace-2-3-4-5} is a straight and so is {\tt
10-Jack-Queen-King-Ace}, but {\tt Queen-King-Ace-2-3} is not.)
\vspace{-0.05in}

\item[flush:] five cards with the same suit
\vspace{-0.05in}

\item[full house:] three cards with one rank, two cards with another
\vspace{-0.05in}

\item[four of a kind:] four cards with the same rank
\vspace{-0.05in}

\item[straight flush:] five cards in sequence (as defined above) and
with the same suit
\vspace{-0.05in}

\end{description}
%
The goal of these exercises is to estimate
the probability of drawing these various hands.

\begin{enumerate}

\item Download the following files from \url{http://thinkpython.com/code}:

\begin{description}

\item[{\tt Card.py}]: A complete version of the {\tt Card},
{\tt Deck} and {\tt Hand} classes in this chapter.

\item[{\tt PokerHand.py}]: An incomplete implementation of a class
that represents a poker hand, and some code that tests it.

\end{description}
%
\item If you run {\tt PokerHand.py}, it deals seven 7-card poker hands
and checks to see if any of them contains a flush.  Read this
code carefully before you go on.

\item Add methods to {\tt PokerHand.py} named \verb"has_pair",
\verb"has_twopair", etc. that return True or False according to
whether or not the hand meets the relevant criteria.  Your code should
work correctly for ``hands'' that contain any number of cards
(although 5 and 7 are the most common sizes).

\item Write a method named {\tt classify} that figures out
the highest-value classification for a hand and sets the
{\tt label} attribute accordingly.  For example, a 7-card hand
might contain a flush and a pair; it should be labeled ``flush''.

\item When you are convinced that your classification methods are
working, the next step is to estimate the probabilities of the various
hands.  Write a function in {\tt PokerHand.py} that shuffles a deck of
cards, divides it into hands, classifies the hands, and counts the
number of times various classifications appear.

\item Print a table of the classifications and their probabilities.
Run your program with larger and larger numbers of hands until the
output values converge to a reasonable degree of accuracy.  Compare
your results to the values at \url{http://en.wikipedia.org/wiki/Hand_rankings}.

\end{enumerate}

Solution: \url{http://thinkpython.com/code/PokerHandSoln.py}.
\end{exercise}


\begin{exercise}
\index{Swampy}
\index{TurtleWorld}

This exercise uses TurtleWorld from Chapter~\ref{turtlechap}.
You will write code that makes Turtles play tag.  If you
are not familiar with the rules of tag, see
\url{http://en.wikipedia.org/wiki/Tag_(game)}.

\begin{enumerate}

\item Download \url{http://thinkpython.com/code/Wobbler.py} and run it.  You
should see a TurtleWorld with three Turtles.  If you press the
{\sf Run} button, the Turtles wander at random.

\item Read the code and make sure you understand how it works.
The {\tt Wobbler} class inherits from {\tt Turtle}, which means
that the {\tt Turtle} methods {\tt lt}, {\tt rt}, {\tt fd}
and {\tt bk} work on Wobblers.

The {\tt step} method gets invoked by TurtleWorld.  It invokes
{\tt steer}, which turns the Turtle in the desired direction,
{\tt wobble}, which makes a random turn in proportion to the Turtle's
clumsiness, and {\tt move}, which moves forward a few pixels,
depending on the Turtle's speed.
\index{Tagger}

\item Create a file named {\tt Tagger.py}.  Import everything from
  {\tt Wobbler}, then define a class named {\tt Tagger} that inherits
  from {\tt Wobbler}.  Call \verb"make_world" passing the {\tt
    Tagger} class object as an argument.

\item Add a {\tt steer} method to {\tt Tagger} to override the one in
  {\tt Wobbler}.  As a starting place, write a version that always
  points the Turtle toward the origin.  Hint: use the math function
  {\tt atan2} and the Turtle attributes {\tt x}, {\tt y} and
  {\tt heading}.

\item Modify {\tt steer} so that the Turtles stay in bounds.
  For debugging, you might want to use the {\sf Step} button,
  which invokes {\tt step} once on each Turtle.

\item Modify {\tt steer} so that each Turtle points toward its nearest
  neighbor.  Hint: Turtles have an attribute, {\tt world}, that is a
  reference to the TurtleWorld they live in, and the TurtleWorld has
  an attribute, {\tt animals}, that is a list of all Turtles in the
  world.

\item Modify {\tt steer} so the Turtles play tag.  You can add methods
  to {\tt Tagger} and you can override {\tt steer} and
  \verb"__init__", but you may not modify or override {\tt step}, {\tt
    wobble} or {\tt move}.  Also, {\tt steer} is allowed to change the
  heading of the Turtle but not the position.

Adjust the rules and your {\tt steer} method for good quality play;
for example, it should be possible for the slow Turtle to tag the
faster Turtles eventually.

\end{enumerate}

Solution: \url{http://thinkpython.com/code/Tagger.py}.
\end{exercise}



\chapter{Case study: Tkinter}
\label{tkinter}

\section{GUI}

Most of the programs we have seen so far are text-based, but
many programs use {\bf graphical user interfaces}, also
known as {\bf GUIs}.
\index{GUI}
\index{graphical user interface}
\index{Tkinter}

Python provides several choices for writing GUI-based programs,
including wxPython, Tkinter, and Qt.  Each has pros and cons, which
is why Python has not converged on a standard.

The one I will present in this chapter is Tkinter because I think
it is the easiest to get started with.  Most of the concepts
in this chapter apply to the other GUI modules, too.

There are several books and web pages about Tkinter.  One of
the best online resources is {\em An Introduction to Tkinter}
by Fredrik Lundh.
\index{Gui module}
\index{module!Gui}
\index{Swampy}

I have written a module called {\tt Gui.py} that comes with
Swampy.  It provides a simplified interface to the functions
and classes in Tkinter.  The examples in this chapter are
based on this module.

Here is a simple example that creates and displays a Gui:

To create a GUI, you have to import {\tt Gui} from Swampy:
%
\begin{verbatim}
from swampy.Gui import *
\end{verbatim}
%
Or, depending on how you installed Swampy, like this:
%
\begin{verbatim}
from Gui import *
\end{verbatim}
%
Then instantiate a Gui object:
%
\begin{verbatim}
g = Gui()
g.title('Gui')
g.mainloop()
\end{verbatim}
%
When you run this code, a window should appear with an empty gray
square and the title {\sf Gui}.  {\tt mainloop} runs the {\bf event
  loop}, which waits for the user to do something and responds
accordingly.  It is an infinite loop; it runs until the user closes
the window, or presses Control-C, or does something that causes the
program to quit.
\index{event loop}
\index{loop!event}
\index{infinite loop}
\index{loop!infinite}

This Gui doesn't do much because it doesn't have any
{\bf widgets}.  Widgets are the elements that make up a
GUI; they include:
\index{widget}

\begin{description}

\item[Button:] A widget, containing text or an image, that
performs an action when pressed.

\item[Canvas:] A region that can display lines, rectangles,
circles and other shapes.

\item[Entry:] A region where users can type text.

\item[Scrollbar:] A widget that controls the visible part of another
widget.

\item[Frame:] A container, often invisible, that contains other
widgets.

\end{description}

The empty gray square you see when you create a Gui is
a Frame.  When you create a new widget, it is added to this Frame.



\section{Buttons and callbacks}
\index{Button widget}
\index{widget!Button}

The method {\tt bu} creates a Button widget:

\begin{verbatim}
button = g.bu(text='Press me.')
\end{verbatim}
%
The return value from {\tt bu} is a Button object.  The button
that appears in the Frame is a graphical representation of this
object; you can control the button by invoking methods on it.
\index{option}

{\tt bu} takes up to 32 parameters that control the appearance
and function of the button.  These parameters are called
{\bf options}.  Instead of providing values for all 32 options,
you can use keyword arguments, like \verb"text='Press me.'",
to specify only the options you need and use the default
values for the rest.
\index{keyword argument}
\index{argument!keyword}

When you add a widget to the Frame, it gets ``shrink-wrapped;''
that is, the Frame shrinks to the size of the Button.  If you
add more widgets, the Frame grows to accommodate them.
\index{Label widget}
\index{widget!Label}

The method {\tt la} creates a Label widget:

\begin{verbatim}
label = g.la(text='Press the button.')
\end{verbatim}
%
By default, Tkinter stacks the widgets top-to-bottom and centers
them.  We'll see how to override that behavior soon.

If you press the button, you will see that it doesn't do much.
That's because you haven't ``wired it up;'' that is, you haven't
told it what to do!

The option that controls the behavior of a button is {\tt command}.
The value of {\tt command} is a function that gets executed when
the button is pressed.  For example, here is a function that creates
a new Label:

\begin{verbatim}
def make_label():
    g.la(text='Thank you.')
\end{verbatim}
%
Now we can create a button with this function as its command:

\begin{verbatim}
button2 = g.bu(text='No, press me!', command=make_label)
\end{verbatim}
%
When you press this button, it should execute \verb"make_label"
and a new label should appear.
\index{callback}

The value of the {\tt command} option
is a function object, which is known as a {\bf callback} because
after you call {\tt bu} to create the button, the flow of execution
``calls back'' when the user presses the button.
\index{event-driven programming}

This kind of flow is characteristic of {\bf event-driven programming}.
User actions, like button presses and key strokes, are called {\bf
events}.  In event-driven programming, the flow of execution is
determined by user actions rather than by the programmer.

The challenge of event-driven programming is to construct a set of
widgets and callbacks that work correctly (or at least generate
appropriate error messages) for any sequence of user actions.

\begin{exercise}

Write a program that creates a GUI with a single button.  When the
button is pressed it should create a second button.  When
{\em that} button is pressed, it should create a label that
says, ``Nice job!''.

What happens if you press the buttons more than once?
Solution: \url{http://thinkpython.com/code/button_demo.py}

\end{exercise}


\section{Canvas widgets}
\index{Canvas widget}
\index{widget!Canvas}

One of the most versatile widgets is the Canvas, which creates
a region for drawing lines, circles and other shapes.  If you
did Exercise~\ref{canvas} you are already familiar with canvases.

The method {\tt ca} creates a new Canvas:

\begin{verbatim}
canvas = g.ca(width=500, height=500)
\end{verbatim}
%
{\tt width} and {\tt height} are the dimensions of the canvas
in pixels.
\index{config method}
\index{method!config}

After you create a widget, you can still change the values of
the options with the
{\tt config} method.  For example, the {\tt bg} option changes
the background color:

\begin{verbatim}
canvas.config(bg='white')
\end{verbatim}
%
The value of {\tt bg} is a string
that names a color.  The set of legal color names is different
for different implementations of Python, but all implementations
provide at least:

\begin{verbatim}
white   black
red     green    blue
cyan    yellow   magenta
\end{verbatim}
%
Shapes on a Canvas are called {\bf items}.  For example,
the Canvas method {\tt circle} draws (you guessed it) a circle:
\index{Canvas item}
\index{item!Canvas}

\begin{verbatim}
item = canvas.circle([0,0], 100, fill='red')
\end{verbatim}
%
The first argument is a coordinate pair that specifies the
center of the circle; the second is the radius.
\index{Canvas coordinate}
\index{coordinate!Canvas}

{\tt Gui.py} provides a standard Cartesian coordinate system with
the origin at the center of the Canvas and the positive $y$ axis
pointing up.  This is different from some other graphics systems
where the origin is in the upper left corner, with the $y$ axis
pointing down.

The {\tt fill} option specifies that the circle should be filled
in with red.

The return value from {\tt circle} is an Item object that
provides methods for modifying the item on the canvas.  For
example, you can use {\tt config} to change any of the circle's
options:

\begin{verbatim}
item.config(fill='yellow', outline='orange', width=10)
\end{verbatim}
%
{\tt width} is the thickness of the outline in pixels;
{\tt outline} is the color.

\begin{exercise}
\label{circle}

Write a program that creates a Canvas and a Button.  When the
user presses the Button, it should draw a circle on the canvas.

\end{exercise}


\section{Coordinate sequences}
\index{coordinate sequence}
\index{sequence!coordinate}

The {\tt rectangle} method takes a sequence of coordinates that
specify opposite corners of the rectangle.  This example
draws a blue rectangle with the lower left corner at the origin
and the upper right corner at $(200,100)$:

\begin{verbatim}
canvas.rectangle([[0, 0], [200, 100]],
                 fill='blue', outline='orange', width=10)
\end{verbatim}
%
This way of specifying corners is called
a {\bf bounding box} because the two points
bound the rectangle.
\index{bounding box}

{\tt oval} takes a bounding box and draws an oval
within the specified rectangle:

\begin{verbatim}
canvas.oval([[0, 0], [200, 100]], outline='orange', width=10)
\end{verbatim}
%
{\tt line} takes a sequence of coordinates and draws
a line that connects the points.  This example draws two legs
of a triangle:

\begin{verbatim}
canvas.line([[0, 100], [100, 200], [200, 100]], width=10)
\end{verbatim}
%
{\tt polygon} takes the same arguments, but it draws the last
leg of the polygon (if necessary) and fills it in:

\begin{verbatim}
canvas.polygon([[0, 100], [100, 200], [200, 100]],
               fill='red', outline='orange', width=10)
\end{verbatim}
%


\section{More widgets}
\index{Text widget}
\index{widget!Text}

Tkinter provides two widgets that let users type text: an
Entry, which is a single line, and a Text widget, which has
multiple lines.
\index{Entry widget}
\index{widget!Entry}

{\tt en} creates a new Entry:

\begin{verbatim}
entry = g.en(text='Default text.')
\end{verbatim}
%
The {\tt text} option allows you to put text into the entry
when it is created.  The {\tt get} method returns the contents
of the Entry (which may have been changed by the user):

\begin{verbatim}
>>> entry.get()
'Default text.'
\end{verbatim}
%
{\tt te} creates a Text widget:

\begin{verbatim}
text = g.te(width=100, height=5)
\end{verbatim}
%
{\tt width} and {\tt height} are the dimensions of the
widget in characters and lines.

{\tt insert} puts text into the Text widget:

\begin{verbatim}
text.insert(END, 'A line of text.')
\end{verbatim}
%
{\tt END} is a special index that indicates the last character in the
Text widget.

You can also specify a character using a dotted index, like {\tt 1.1},
which has the line number before the dot and the column number after.
The following example adds the letters \verb"'nother'" after the first
character of the first line.

\begin{verbatim}
>>> text.insert(1.1, 'nother')
\end{verbatim}
%
The {\tt get} method reads the text in the widget; it takes a start
and end index as arguments.  The following example returns all the
text in the widget, including the newline character:

\begin{verbatim}
>>> text.get(0.0, END)
'Another line of text.\n'
\end{verbatim}
%
The {\tt delete} method removes text from the widget;
the following example deletes all but the first two characters:

\begin{verbatim}
>>> text.delete(1.2, END)
>>> text.get(0.0, END)
'An\n'
\end{verbatim}
%

\begin{exercise}
\label{circle2}

Modify your solution to Exercise~\ref{circle} by adding an
Entry widget and a second button.  When the user presses the
second button, it should read a color name from the Entry and
use it to change the fill color of the circle.  Use {\tt config}
to modify the existing circle; don't create a new one.

Your program should handle the case where the user tries to
change the color of a circle that hasn't been created, and
the case where the color name is invalid.

You can see my solution at \url{http://thinkpython.com/code/circle_demo.py}.

\end{exercise}


\section{Packing widgets}

So far we have been stacking widgets in a single column, but in most
GUIs the layout is more complicated.  For example,
Figure~\ref{fig.turtleworld} shows a simplified version of
TurtleWorld (see Chapter~\ref{turtlechap}).

\begin{figure}
\centerline{\includegraphics[scale=0.5]{figs/TurtleWorld.pdf}}
\caption{TurtleWorld after running the snowflake code.}
\label{fig.turtleworld}
\end{figure}


This section presents the code that creates this GUI, broken into a
series of steps.  You can download the complete example
from \url{http://thinkpython.com/code/SimpleTurtleWorld.py}.

At the top level, this GUI contains two widgets---a Canvas and a
Frame---arranged in a row.  So the first step is to create the row.
\index{SimpleTurtleWorld class}
\index{class!SimpleTurtleWorld}

\begin{verbatim}
class SimpleTurtleWorld(TurtleWorld):
    """This class is identical to TurtleWorld, but the code that
    lays out the GUI is simplified for explanatory purposes."""

    def setup(self):
        self.row()
        ...
\end{verbatim}
%
{\tt setup} is the function that creates and arranges the widgets.
Arranging widgets in a GUI is called {\bf packing}.
\index{packing widgets}
\index{widget, packing}
\index{Frame widget}
\index{widget!Frame}

{\tt row} creates a row Frame and makes it the ``current Frame.''
Until this Frame is closed or another Frame is created, all
subsequent widgets are packed in a row.

Here is the code that creates the Canvas and the column Frame
that hold the other widgets:

\begin{verbatim}
        self.canvas = self.ca(width=400, height=400, bg='white')
        self.col()
\end{verbatim}
%
The first widget in the column is a grid Frame, which contains
four buttons arranged two-by-two:

\begin{verbatim}
        self.gr(cols=2)
        self.bu(text='Print canvas', command=self.canvas.dump)
        self.bu(text='Quit', command=self.quit)
        self.bu(text='Make Turtle', command=self.make_turtle)
        self.bu(text='Clear', command=self.clear)
        self.endgr()
\end{verbatim}
%
{\tt gr} creates the grid; the argument is the number of
columns.  Widgets in the grid are
laid out left-to-right, top-to-bottom.
\index{callback}
\index{bound method}
\index{method, bound}
\index{subject}

The first button uses {\tt self.canvas.dump} as a callback; the second
uses {\tt self.quit}.  These are {\bf bound methods}, which means they
are associated with a particular object.  When they are invoked, they
are invoked on the object.

The next widget in the column is a row Frame that contains
a Button and an Entry:

\begin{verbatim}
        self.row([0,1], pady=30)
        self.bu(text='Run file', command=self.run_file)
        self.en_file = self.en(text='snowflake.py', width=5)
        self.endrow()
\end{verbatim}
%
The first argument to {\tt row} is a list of weights that
determines how extra space is allocated between widgets.
The list {\tt [0,1]} means that all extra space is allocated
to the second widget, which is the Entry.  If you run this code
and resize the window, you will see that the Entry grows and
the Button doesn't.

The option {\tt pady} ``pads'' this row in the $y$ direction,
adding 30 pixels of space above and below.

{\tt endrow} ends this row of widgets, so subsequent widgets are
packed in the column Frame.  {\tt Gui.py} keeps a stack of Frames:

\begin{itemize}

\item When you use {\tt row}, {\tt col} or {\tt gr} to create a Frame,
it goes on top of the stack and becomes the current Frame.

\item When you use {\tt endrow}, {\tt endcol} or {\tt endgr} to close
a Frame, it gets popped off the stack and the previous Frame on the
stack becomes the current Frame.

\end{itemize}

The method \verb"run_file" reads the contents of the Entry,
uses it as a filename, reads the contents
and passes it to \verb"run_code".  {\tt self.inter} is an
Interpreter object that knows how to take a string and
execute it as Python code.

\begin{verbatim}
    def run_file(self):
        filename = self.en_file.get()
        fp = open(filename)
        source = fp.read()
        self.inter.run_code(source, filename)
\end{verbatim}
%
The last two widgets are a Text widget and a Button:

\begin{verbatim}
        self.te_code = self.te(width=25, height=10)
        self.te_code.insert(END, 'world.clear()\n')
        self.te_code.insert(END, 'bob = Turtle(world)\n')

        self.bu(text='Run code', command=self.run_text)
\end{verbatim}
%
\verb"run_text" is similar to \verb"run_file" except that it takes
the code from the Text widget instead of from a file:

\begin{verbatim}
    def run_text(self):
        source = self.te_code.get(1.0, END)
        self.inter.run_code(source, '<user-provided code>')
\end{verbatim}
%
Unfortunately, the details of widget layout are different in
other languages, and in different Python modules.
Tkinter alone provides three different mechanisms for arranging
widgets.  These mechanisms are called {\bf geometry managers}.
The one I demonstrated in this section is the ``grid'' geometry
manager; the others are called ``pack'' and ``place''.
\index{geometry manager}

Fortunately, most of the concepts in this section apply to
other GUI modules and other languages.


\section{Menus and Callables}
\index{Menubutton widget}
\index{widget!Menubutton}

A Menubutton is a widget that looks like a button, but when pressed
it pops up a menu.  After the user selects an item, the menu
disappears.

Here is code that creates a color selection Menubutton
(you can download it from \url{http://thinkpython.com/code/menubutton_demo.py}):

\begin{verbatim}
g = Gui()
g.la('Select a color:')
colors = ['red', 'green', 'blue']
mb = g.mb(text=colors[0])
\end{verbatim}
%
{\tt mb} creates the Menubutton.  Initially, the text on the button is
the name of the default color.  The following loop creates one menu
item for each color:

\begin{verbatim}
for color in colors:
    g.mi(mb, text=color, command=Callable(set_color, color))
\end{verbatim}
%
The first argument of {\tt mi} is the Menubutton these items are
associated with.
\index{callback}
\index{Callable object}
\index{object!Callable}

The {\tt command} option is a Callable object, which is something new.
So far we have seen functions and bound methods used as callbacks,
which works fine if you don't have to pass any arguments to
the function.  Otherwise you have to construct a Callable object
that contains a function, like \verb"set_color", and its arguments,
like {\tt color}.

The Callable object stores a reference to the function and the
arguments as attributes.  Later, when the user clicks on a menu
item, the callback calls the function and passes the stored
arguments.

Here is what \verb"set_color" might look like:

\begin{verbatim}
def set_color(color):
    mb.config(text=color)
    print color
\end{verbatim}
%
When the user selects a menu item and \verb"set_color" is called,
it configures the Menubutton to display the newly-selected color.
It also print the color; if you try this example, you can confirm that
\verb"set_color" is called when you select an item (and {\em not}
called when you create the Callable object).


\section{Binding}
\index{binding}
\index{callback}

A {\bf binding} is an association between a widget, an event and a
callback: when an event (like a button press) happens on a widget, the
callback is invoked.

Many widgets have default bindings.  For example, when you press
a button, the default binding changes the relief of the button
to make it look depressed.  When you release the button, the
binding restores the appearance of the button and invokes the
callback specified with the {\tt command} option.

You can use the {\tt bind} method to override these default
bindings or to add new ones.  For example, this code creates a
binding for a canvas (you can download the code in this
section from \url{http://thinkpython.com/code/draggable_demo.py}):

\begin{verbatim}
ca.bind('<ButtonPress-1>', make_circle)
\end{verbatim}
%
The first argument is an event string; this event is triggered
when the user presses the left mouse button.  Other mouse
events include {\tt ButtonMotion}, {\tt ButtonRelease} and
{\tt Double-Button}.
\index{event string}
\index{event handler}

The second argument is an event handler.  An event handler
is a function or bound method, like a callback, but an important
difference is that an event handler takes an Event object as a
parameter.  Here is an example:

\begin{verbatim}
def make_circle(event):
    pos = ca.canvas_coords([event.x, event.y])
    item = ca.circle(pos, 5, fill='red')
\end{verbatim}
%
The Event object contains information about the type of event and
details like the coordinates of the mouse pointer.  In this example
the information we need is
the location of the mouse click.  These
values are in ``pixel coordinates,'' which are defined by the
underlying graphical system.  The method \verb"canvas_coords"
translates them to ``Canvas coordinates,'' which are compatible with
Canvas methods like {\tt circle}.
\index{Event object}
\index{object!Event}

For Entry widgets, it is common to bind the \verb"<Return>" event,
which is triggered when the user presses the {\sf Return} or
{\sf Enter} key.  For example, the following code creates a Button
and an Entry.

\begin{verbatim}
bu = g.bu('Make text item:', make_text)
en = g.en()
en.bind('<Return>', make_text)
\end{verbatim}
%
\verb"make_text" is called when the Button is pressed or when
the user hits {\sf Return} while typing in the Entry.  To make
this work, we need a function that can be called as a command
(with no arguments) or as an event handler (with an Event
as an argument):

\begin{verbatim}
def make_text(event=None):
    text = en.get()
    item = ca.text([0,0], text)
\end{verbatim}
%
\verb"make_text" gets the contents of the Entry and displays
it as a Text item in the Canvas.

It is also possible to create bindings for Canvas items.
The following is a class definition for {\tt Draggable},
which is a child class of {\tt Item} that provides bindings
that implement drag-and-drop capability.
\index{drag-and-drop}

\begin{verbatim}
class Draggable(Item):

    def __init__(self, item):
        self.canvas = item.canvas
        self.tag = item.tag
        self.bind('<Button-3>', self.select)
        self.bind('<B3-Motion>', self.drag)
        self.bind('<Release-3>', self.drop)
\end{verbatim}
%
The init method takes an Item as a parameter.  It copies
the attributes of the Item and then creates bindings for
three events: a button press, button motion, and button release.

The event handler {\tt select} stores the coordinates
of the current event and the original color of the item, then
changes the color to yellow:

\begin{verbatim}
    def select(self, event):
        self.dragx = event.x
        self.dragy = event.y

        self.fill = self.cget('fill')
        self.config(fill='yellow')
\end{verbatim}
%
{\tt cget} stands for ``get configuration;'' it takes the name of an
option as a string and returns the current value of that option.

{\tt drag} computes how far the object has moved relative to the
starting place, updates the stored coordinates, and then moves the
item.
\index{update!coordinate}

\begin{verbatim}
    def drag(self, event):
        dx = event.x - self.dragx
        dy = event.y - self.dragy

        self.dragx = event.x
        self.dragy = event.y

        self.move(dx, dy)
\end{verbatim}
%
This computation is done in pixel coordinates; there is no need to
convert to Canvas coordinates.
\index{Canvas coordinate}
\index{coordinate!Canvas}
\index{pixel coordinate}
\index{coordinate!pixel}

Finally, {\tt drop} restores the original color of the item:

\begin{verbatim}
    def drop(self, event):
        self.config(fill=self.fill)
\end{verbatim}
%
You can use the {\tt Draggable} class to add drag-and-drop
capability to an existing item.  For example, here is a modified
version of \verb"make_circle" that uses {\tt circle} to create
an Item and {\tt Draggable} to make it draggable:

\begin{verbatim}
def make_circle(event):
    pos = ca.canvas_coords([event.x, event.y])
    item = ca.circle(pos, 5, fill='red')
    item = Draggable(item)
\end{verbatim}
%
This example demonstrates one of the benefits of inheritance: you can
modify the capabilities of a parent class without modifying its
definition.  This is particularly useful if you want to change
behavior defined in a module you did not write.


\section{Debugging}
\index{debugging}

One of the challenges of GUI programming is keeping track of
which things happen while the GUI is being built and which
things happen later in response to user events.
\index{callback}

For example, when you are setting up a callback, it is a common error
to call the function rather than passing a reference to it:

\begin{verbatim}
def the_callback():
    print 'Called.'

g.bu(text='This is wrong!', command=the_callback())
\end{verbatim}
%
If you run this code, you will see that it calls \verb"the_callback"
immediately, and {\em then} creates the button.  When you press the
button, it does nothing because the return value from
\verb"the_callback" is {\tt None}.
Usually you do not want to invoke a callback while you are
setting up the GUI; it should only be invoked later in response to
a user event.
\index{flow of execution}
\index{event-driven programming}

Another challenge of GUI programming is that you don't have control
of the flow of execution.  Which parts of the program execute
and their order are determined by user actions.
That means that you have to design your program to work correctly
for any possible sequence of events.

For example, the GUI in Exercise~\ref{circle2} has two widgets:
one creates a Circle item and the other changes the color of the
Circle.  If the user creates the circle and then changes its color,
there's no problem.  But what if the user changes the color of
a circle that doesn't exist yet?  Or creates more than one circle?

As the number of widgets grows, it is increasingly difficult to
imagine all possible sequences of events.  One way to manage this
complexity is to encapsulate the state of the system in an object
and then consider:

\begin{itemize}

\item What are the possible states?  In the Circle example, we
might consider two states: before and after the user creates the
first circle.

\item In each state, what events can occur?  In the example,
the user can press either of the buttons, or quit.

\item For each state-event pair, what is the desired outcome?
Since there are two states and two buttons, there are four
state-event pairs to consider.

\item What can cause a transition from one state to another?
In this case, there is a transition when the user creates
the first circle.

\end{itemize}

You might also find it useful to define, and check, invariants that
should hold regardless of the sequence of events.
\index{invariant}

This approach to GUI programming can help you write correct
code without taking the time to test every possible sequence
of user events!


\section{Glossary}

\begin{description}

\item[GUI:] A graphical user interface.
\index{GUI}

\item[widget:] One of the elements that makes up a GUI, including
buttons, menus, text entry fields, etc.
\index{widget}

\item[option:] A value that controls the appearance or function of
a widget.
\index{option}

\item[keyword argument:] An argument that indicates the parameter
name as part of the function call.
\index{keyword argument}

\item[callback:] A function associated with a widget that is
called when the user performs an action.
\index{callback}

\item[bound method:] A method associated with a particular instance.
\index{bound method}

\item[event-driven programming:] A style of programming in which
the flow of execution is determined by user actions.
\index{event-driven programming}

\item[event:] A user action, like a mouse click or key press, that
causes a GUI to respond.
\index{event}

\item[event loop:] An infinite loop that waits for user actions
and responds.
\index{event loop}

\item[item:] A graphical element on a Canvas widget.
\index{item!Canvas}

\item[bounding box:] A rectangle that encloses a set of items,
usually specified by two opposing corners.
\index{bounding box}

\item[pack:] To arrange and display the elements of a GUI.
\index{packing widgets}

\item[geometry manager:] A system for packing widgets.
\index{geometry manager}

\item[binding:] An association between a widget, an event, and
an event handler.  The event handler is called when the event
occurs in the widget.
\index{binding}

\end{description}


\section{Exercises}

\begin{exercise}
\index{image viewer}

For this exercise, you will write an image viewer.  Here is
a simple example:

\begin{verbatim}
g = Gui()
canvas = g.ca(width=300)
photo = PhotoImage(file='danger.gif')
canvas.image([0,0], image=photo)
g.mainloop()
\end{verbatim}
%
{\tt PhotoImage} reads a file and returns a {\tt PhotoImage} object
that Tkinter can display.  {\tt Canvas.image} puts the image on the
canvas, centered on the given coordinates.  You can also put images on
labels, buttons, and some other widgets:

\begin{verbatim}
g.la(image=photo)
g.bu(image=photo)
\end{verbatim}
%
PhotoImage can only handle a few image formats, like GIF and PPM,
but we can use the Python Imaging Library (PIL) to read other
files.
\index{Python Imaging Library (PIL)}
\index{PIL (Python Imaging Library)}
\index{Image module}
\index{module!Image}

The name of the PIL module is {\tt Image}, but Tkinter defines an
object with the same name.  To avoid the conflict, you can use {\tt
  import...as} like this:

\begin{verbatim}
import Image as PIL
import ImageTk
\end{verbatim}
%
The first line imports {\tt Image} and
gives it the local name {\tt PIL}.  The second
line imports {\tt ImageTk}, which can translate a PIL
image into a Tkinter PhotoImage.  Here's an example:

\begin{verbatim}
image = PIL.open('allen.png')
photo2 = ImageTk.PhotoImage(image)
g.la(image=photo2)
\end{verbatim}
%

\begin{enumerate}

\item Download \verb"image_demo.py", \verb"danger.gif" and \verb"allen.png"
from \url{http://thinkpython.com/code}.  Run \verb"image_demo.py".  You
might have to install {\tt PIL} and {\tt ImageTk}.
They are probably in your software repository,  but if not
you can get them from \url{http://pythonware.com/products/pil}.

\item In \verb"image_demo.py" change the name of the second
PhotoImage from {\tt photo2} to {\tt photo} and run the program
again.  You should see the second PhotoImage but not the first.

The problem is that when you reassign {\tt photo} it overwrites
the reference to the first PhotoImage, which then disappears.  The
same thing happens if you assign a PhotoImage to a local
variable; it disappears when the function ends.

To avoid this problem, you have to store a reference to each
PhotoImage you want to keep.  You can use a global variable, or
store PhotoImages in a data structure or as an attribute of
an object.

This behavior can be frustrating, which is why I am warning
you (and why the example image says ``Danger!'').
\index{bug!worst ever}
\index{worst bug!ever}

\item Starting with this example, write a program that takes
the name of a directory and loops through all the files, displaying
any files that PIL recognizes as images.  You can use a {\tt try}
statement to catch the files PIL doesn't recognize.

When the user clicks on the image, the program should display the next one.

\item PIL provides a variety of methods for manipulating images.
You can read about them at \url{http://pythonware.com/library/pil/handbook}.
As a challenge, choose a few of these methods and provide a
GUI for applying them to images.

\end{enumerate}

Solution: \url{http://thinkpython.com/code/ImageBrowser.py}.

\end{exercise}


\begin{exercise}
\index{vector graphics}
\index{SVG}

A vector graphics editor is a program that allows users to draw and
edit shapes on the screen and generate output files in vector graphics
formats like Postscript and SVG.

Write a simple vector graphics editor using Tkinter.  At a
minimum, it should allow users to draw lines, circles and
rectangles, and it should use {\tt Canvas.dump} to
generate a Postscript description of the contents of the
Canvas.

As a challenge, you could allow users to select and resize
items on the Canvas.

% TODO: write a solution!

\end{exercise}


\begin{exercise}

Use Tkinter to write a basic web browser.  It
should have a Text widget where the user can enter a URL
and a Canvas to display the contents of the page.
\index{urllib module}
\index{module!urllib}
\index{URL}
\index{HTMLParser module}
\index{module!HTMLParser}

You can use the {\tt urllib} module to download files
(see Exercise~\ref{urllib}) and
the {\tt HTMLParser} module to parse the HTML
tags (see \url{http://docs.python.org/2/library/htmlparser.html}).
\index{plain text}
\index{text!plain}
\index{hyperlink}

At a minimum your browser should handle plain text and hyperlinks.  As
a challenge you could handle background colors, text
formatting tags and images.

% TODO: write a solution!

\end{exercise}



\appendix

\chapter{Debugging}

\index{debugging}
Different kinds of errors can occur
in a program, and it is useful to distinguish among them
in order to track them down more quickly:

\begin{itemize}

\item Syntax errors are produced by Python when it is translating the
  source code into byte code.  They usually indicate that there is
  something wrong with the syntax of the program.  Example: Omitting
  the colon at the end of a {\tt def} statement yields the somewhat
  redundant message {\tt SyntaxError: invalid syntax}.

\item Runtime errors are produced by the interpreter if something goes
  wrong while the program is running.  Most runtime error messages
  include information about where the error occurred and what
  functions were executing.  Example: An infinite recursion eventually
  causes the runtime error ``maximum recursion depth exceeded.''

\item Semantic errors are problems with a program that runs without
  producing error messages but doesn't do the right thing.  Example:
  An expression may not be evaluated in the order you expect, yielding
  an incorrect result.

\end{itemize}
\index{syntax error}
\index{runtime error}
\index{semantic error}
\index{error!compile-time}
\index{error!syntax}
\index{error!runtime}
\index{error!semantic}
\index{exception}

The first step in debugging is to figure out which kind of
error you are dealing with.  Although the following sections are
organized by error type, some techniques are
applicable in more than one situation.


\section{Syntax errors}
\index{error message}

Syntax errors are usually easy to fix once you figure out what they
are.  Unfortunately, the error messages are often not helpful.
The most common messages are {\tt SyntaxError: invalid syntax} and
{\tt SyntaxError: invalid token}, neither of which is very informative.

On the other hand, the message does tell you where in the program the
problem occurred.  Actually, it tells you where Python
noticed a problem, which is not necessarily where the error
is.  Sometimes the error is prior to the location of the error
message, often on the preceding line.
\index{incremental development}
\index{development plan!incremental}

If you are building the program incrementally, you should have
a good idea about where the error is.  It will be in the last
line you added.

If you are copying code from a book, start by comparing
your code to the book's code very carefully.  Check every character.
At the same time, remember that the book might be wrong, so
if you see something that looks like a syntax error, it might be.

Here are some ways to avoid the most common syntax errors:
\index{syntax}

\begin{enumerate}

\item Make sure you are not using a Python keyword for a variable name.
\index{keyword}

\item Check that you have a colon at the end of the header of every
compound statement, including {\tt for}, {\tt while},
{\tt if}, and {\tt def} statements.
\index{header}
\index{colon}

\item Make sure that any strings in the code have matching
quotation marks.
\index{quotation mark}

\item If you have multiline strings with triple quotes (single or double), make
sure you have terminated the string properly.  An unterminated string
may cause an {\tt invalid token} error at the end of your program,
or it may treat the following part of the program as a string until it
comes to the next string.  In the second case, it might not produce an error
message at all!
\index{multiline string}
\index{string!multiline}

\item An unclosed opening operator---\verb+(+, \verb+{+, or
  \verb+[+---makes Python continue with the next line as part of the
  current statement.  Generally, an error occurs almost immediately in
  the next line.

\item Check for the classic {\tt =} instead of {\tt ==} inside
a conditional.
\index{conditional}

\item Check the indentation to make sure it lines up the way it
is supposed to.  Python can handle space and tabs, but if you mix
them it can cause problems.  The best way to avoid this problem
is to use a text editor that knows about Python and generates
consistent indentation.
\index{indentation}
\index{whitespace}

\end{enumerate}

If nothing works, move on to the next section...


\subsection{I keep making changes and it makes no difference.}

If the interpreter says there is an error and you don't see it, that
might be because you and the interpreter are not looking at the same
code.  Check your programming environment to make sure that the
program you are editing is the one Python is trying to run.

If you are not sure, try putting an obvious and deliberate syntax
error at the beginning of the program.  Now run it again.  If the
interpreter doesn't find the new error, you are not running the
new code.

There are a few likely culprits:

\begin{itemize}

\item You edited the file and forgot to save the changes before
running it again.  Some programming environments do this
for you, but some don't.

\item You changed the name of the file, but you are still running
the old name.

\item Something in your development environment is configured
incorrectly.

\item If you are writing a module and using {\tt import},
make sure you don't give your module the same name as one
of the standard Python modules.
\index{module!reload}
\index{reload function}
\index{function!reload}

\item If you are using {\tt import} to read a module, remember
that you have to restart the interpreter or use {\tt reload}
to read a modified file.  If you import the module again, it
doesn't do anything.

\end{itemize}

If you get stuck and you can't figure out what is going on, one
approach is to start again with a new program like ``Hello, World!,''
and make sure you can get a known program to run.  Then gradually add
the pieces of the original program to the new one.


\section{Runtime errors}

Once your program is syntactically correct,
Python can compile it and at least start running it.  What could
possibly go wrong?


\subsection{My program does absolutely nothing.}

This problem is most common when your file consists of functions and
classes but does not actually invoke anything to start execution.
This may be intentional if you only plan to import this module to
supply classes and functions.

If it is not intentional, make sure that you
are invoking a function to start execution, or execute one from
the interactive prompt.  Also see the ``Flow of Execution'' section
below.


\subsection{My program hangs.}
\index{infinite loop}
\index{infinite recursion}
\index{hanging}

If a program stops and seems to be doing nothing, it is ``hanging.''
Often that means that it is caught in an infinite loop or infinite
recursion.

\begin{itemize}

\item If there is a particular loop that you suspect is the
problem, add a {\tt print} statement immediately before the loop that says
``entering the loop'' and another immediately after that says
``exiting the loop.''

Run the program.  If you get the first message and not the second,
you've got an infinite loop.  Go to the ``Infinite Loop'' section
below.

\item Most of the time, an infinite recursion will cause the program
to run for a while and then produce a ``RuntimeError: Maximum
recursion depth exceeded'' error.  If that happens, go to the
``Infinite Recursion'' section below.

If you are not getting this error but you suspect there is a problem
with a recursive method or function, you can still use the techniques
in the ``Infinite Recursion'' section.

\item If neither of those steps works, start testing other
loops and other recursive functions and methods.

\item If that doesn't work, then it is possible that
you don't understand the flow of execution in your program.
Go to the ``Flow of Execution'' section below.

\end{itemize}


\subsubsection{Infinite Loop}
\index{infinite loop}
\index{loop!infinite}
\index{condition}
\index{loop!condition}

If you think you have an infinite loop and you think you know
what loop is causing the problem, add a {\tt print} statement at
the end of the loop that prints the values of the variables in
the condition and the value of the condition.

For example:

\begin{verbatim}
while x > 0 and y < 0 :
    # do something to x
    # do something to y

    print "x: ", x
    print "y: ", y
    print "condition: ", (x > 0 and y < 0)
\end{verbatim}
%
Now when you run the program, you will see three lines of output
for each time through the loop.  The last time through the
loop, the condition should be {\tt false}.  If the loop keeps
going, you will be able to see the values of {\tt x} and {\tt y},
and you might figure out why they are not being updated correctly.


\subsubsection{Infinite Recursion}
\index{infinite recursion}
\index{recursion!infinite}

Most of the time, an infinite recursion will cause the program to run
for a while and then produce a {\tt Maximum recursion depth exceeded}
error.

If you suspect that a function or method is causing an infinite
recursion, start by checking to make sure that there is a base case.
In other words, there should be some condition that will cause the
function or method to return without making a recursive invocation.
If not, then you need to rethink the algorithm and identify a base
case.

If there is a base case but the program doesn't seem to be reaching
it, add a {\tt print} statement at the beginning of the function or method
that prints the parameters.  Now when you run the program, you will see
a few lines of output every time the function or method is invoked,
and you will see the parameters.  If the parameters are not moving
toward the base case, you will get some ideas about why not.


\subsubsection{Flow of Execution}
\index{flow of execution}

If you are not sure how the flow of execution is moving through
your program, add {\tt print} statements to the beginning of each
function with a message like ``entering function {\tt foo},'' where
{\tt foo} is the name of the function.

Now when you run the program, it will print a trace of each
function as it is invoked.


\subsection{When I run the program I get an exception.}
\index{exception}
\index{runtime error}

If something goes wrong during runtime, Python
prints a message that includes the name of the
exception, the line of the program where the problem occurred,
and a traceback.
\index{traceback}

The traceback identifies the function that is currently running,
and then the function that invoked it, and then the function that
invoked {\em that}, and so on.  In other words, it traces the
sequence of function invocations that got you to where you are.  It
also includes the line number in your file where each of these
calls occurs.

The first step is to examine the place in the program where
the error occurred and see if you can figure out what happened.
These are some of the most common runtime errors:

\begin{description}

\item[NameError:]  You are trying to use a variable that doesn't
exist in the current environment.
Remember that local variables are local.  You
cannot refer to them from outside the function where they are defined.
\index{NameError}
\index{TypeError}
\index{exception!NameError}
\index{exception!TypeError}

\item[TypeError:] There are several possible causes:

\begin{itemize}

\item  You are trying to use a value improperly.  Example: indexing
a string, list, or tuple with something other than an integer.
\index{index}

\item There is a mismatch between the items in a format string and
the items passed for conversion.  This can happen if either the number
of items does not match or an invalid conversion is called for.
\index{format operator}
\index{operator!format}

\item You are passing the wrong number of arguments to a function or method.
For methods, look at the method definition and
check that the first parameter is {\tt self}.  Then look at the
method invocation; make sure you are invoking the method on an
object with the right type and providing the other arguments
correctly.

\end{itemize}

\item[KeyError:]  You are trying to access an element of a dictionary
using a key that the dictionary does not contain.
\index{KeyError}
\index{exception!KeyError}
\index{dictionary}

\item[AttributeError:] You are trying to access an attribute or method
that does not exist.  Check the spelling!  You can use
{\tt dir} to list the attributes that do exist.

If an AttributeError indicates that an object has {\tt NoneType},
that means that it is {\tt None}.  One common cause is forgetting
to return a value from a function; if you get to the end of
a function without hitting a {\tt return} statement, it returns
{\tt None}.  Another common cause is using the result from
a list method, like {\tt sort}, that returns {\tt None}.
\index{AttributeError}
\index{exception!AttributeError}

\item[IndexError:] The index you are using
to access a list, string, or tuple is greater than
its length minus one.  Immediately before the site of the error,
add a {\tt print} statement to display
the value of the index and the length of the array.
Is the array the right size?  Is the index the right value?
\index{IndexError}
\index{exception!IndexError}

\end{description}
\index{debugger (pdb)}
\index{Python debugger (pdb)}
\index{pdb (Python debugger)}

The Python debugger ({\tt pdb}) is useful for tracking down
Exceptions because it allows you to examine the state of the
program immediately before the error.  You can read
about {\tt pdb} at \url{http://docs.python.org/2/library/pdb.html}.


\subsection{I added so many {\tt print} statements I get inundated with
output.}
\index{print statement}
\index{statement!print}

One of the problems with using {\tt print} statements for debugging
is that you can end up buried in output.  There are two ways
to proceed: simplify the output or simplify the program.

To simplify the output, you can remove or comment out {\tt print}
statements that aren't helping, or combine them, or format
the output so it is easier to understand.

To simplify the program, there are several things you can do.  First,
scale down the problem the program is working on.  For example, if you
are searching a list, search a {\em small} list.  If the program takes
input from the user, give it the simplest input that causes the
problem.
\index{dead code}

Second, clean up the program.  Remove dead code and reorganize the
program to make it as easy to read as possible.  For example, if you
suspect that the problem is in a deeply nested part of the program,
try rewriting that part with simpler structure.  If you suspect a
large function, try splitting it into smaller functions and testing them
separately.
\index{testing!minimal test case}
\index{test case, minimal}

Often the process of finding the minimal test case leads you to the
bug.  If you find that a program works in one situation but not in
another, that gives you a clue about what is going on.

Similarly, rewriting a piece of code can help you find subtle
bugs.  If you make a change that you think shouldn't affect the
program, and it does, that can tip you off.


\section{Semantic errors}
\index{semantic error}
\index{error!semantic}

In some ways, semantic errors are the hardest to debug,
because the interpreter provides no information
about what is wrong.  Only you know what the program is supposed to
do.

The first step is to make a connection between the program
text and the behavior you are seeing.  You need a hypothesis
about what the program is actually doing.  One of the things
that makes that hard is that computers run so fast.

You will often wish that you could slow the program down to human
speed, and with some debuggers you can.  But the time it takes to
insert a few well-placed {\tt print} statements is often short compared to
setting up the debugger, inserting and removing breakpoints, and
``stepping'' the program to where the error is occurring.

\subsection{My program doesn't work.}

You should ask yourself these questions:

\begin{itemize}

\item Is there something the program was supposed to do but
which doesn't seem to be happening?  Find the section of the code
that performs that function and make sure it is executing when
you think it should.

\item Is something happening that shouldn't?  Find code in
your program that performs that function and see if it is
executing when it shouldn't.

\item Is a section of code producing an effect that is not
what you expected?  Make sure that you understand the code in
question, especially if it involves invocations to functions or methods in
other Python modules.  Read the documentation for the functions you invoke.
Try them out by writing simple test cases and checking the results.

\end{itemize}

In order to program, you need to have a mental model of how
programs work.  If you write a program that doesn't do what you expect,
very often the problem is not in the program; it's in your mental
model.
\index{model, mental}
\index{mental model}

The best way to correct your mental model is to break the program
into its components (usually the functions and methods) and test
each component independently.  Once you find the discrepancy
between your model and reality, you can solve the problem.

Of course, you should be building and testing components as you
develop the program.  If you encounter a problem,
there should be only a small amount of new code
that is not known to be correct.


\subsection{I've got a big hairy expression and it doesn't
do what I expect.}
\index{expression!big and hairy}
\index{big, hairy expression}

Writing complex expressions is fine as long as they are readable,
but they can be hard to debug.  It is often a good idea to
break a complex expression into a series of assignments to
temporary variables.

For example:

\begin{verbatim}
self.hands[i].addCard(self.hands[self.findNeighbor(i)].popCard())
\end{verbatim}
%
This can be rewritten as:

\begin{verbatim}
neighbor = self.findNeighbor(i)
pickedCard = self.hands[neighbor].popCard()
self.hands[i].addCard(pickedCard)
\end{verbatim}
%
The explicit version is easier to read because the variable
names provide additional documentation, and it is easier to debug
because you can check the types of the intermediate variables
and display their values.
\index{temporary variable}
\index{variable!temporary}
\index{order of operations}
\index{precedence}

Another problem that can occur with big expressions is
that the order of evaluation may not be what you expect.
For example, if you are translating the expression
$\frac{x}{2 \pi}$ into Python, you might write:

\begin{verbatim}
y = x / 2 * math.pi
\end{verbatim}
%
That is not correct because multiplication and division have
the same precedence and are evaluated from left to right.
So this expression computes $x \pi / 2$.

A good way to debug expressions is to add parentheses to make
the order of evaluation explicit:

\begin{verbatim}
 y = x / (2 * math.pi)
\end{verbatim}
%
Whenever you are not sure of the order of evaluation, use
parentheses.  Not only will the program be correct (in the sense
of doing what you intended), it will also be more readable for
other people who haven't memorized the rules of precedence.


\subsection{I've got a function or method that doesn't return what I
expect.}
\index{return statement}
\index{statement!return}

If you have a {\tt return} statement with a complex expression,
you don't have a chance to print the {\tt return} value before
returning.  Again, you can use a temporary variable.  For
example, instead of:

\begin{verbatim}
return self.hands[i].removeMatches()
\end{verbatim}
%
you could write:

\begin{verbatim}
count = self.hands[i].removeMatches()
return count
\end{verbatim}
%
Now you have the opportunity to display the value of
{\tt count} before returning.


\subsection{I'm really, really stuck and I need help.}

First, try getting away from the computer for a few minutes.
Computers emit waves that affect the brain, causing these
symptoms:

\begin{itemize}

\item Frustration and rage.
\index{frustration}
\index{rage}
\index{debugging!emotional response}
\index{emotional debugging}

\item Superstitious beliefs (``the computer hates me'') and
magical thinking (``the program only works when I wear my
hat backward'').
\index{debugging!superstition}
\index{superstitious debugging}

\item Random walk programming (the attempt to program by writing
every possible program and choosing the one that does the right
thing).
\index{random walk programming}
\index{development plan!random walk programming}

\end{itemize}

If you find yourself suffering from any of these symptoms, get
up and go for a walk.  When you are calm, think about the program.
What is it doing?  What are some possible causes of that
behavior?  When was the last time you had a working program,
and what did you do next?

Sometimes it just takes time to find a bug.  I often find bugs
when I am away from the computer and let my mind wander.  Some
of the best places to find bugs are trains, showers, and in bed,
just before you fall asleep.


\subsection{No, I really need help.}

It happens.  Even the best programmers occasionally get stuck.
Sometimes you work on a program so long that you can't see the
error.  A fresh pair of eyes is just the thing.

Before you bring someone else in, make sure you are prepared.
Your program should be as simple
as possible, and you should be working on the smallest input
that causes the error.  You should have {\tt print} statements in the
appropriate places (and the output they produce should be
comprehensible).  You should understand the problem well enough
to describe it concisely.

When you bring someone in to help, be sure to give
them the information they need:

\begin{itemize}

\item If there is an error message, what is it
and what part of the program does it indicate?

\item What was the last thing you did before this error occurred?
What were the last lines of code that you wrote, or what is
the new test case that fails?

\item What have you tried so far, and what have you learned?

\end{itemize}

When you find the bug, take a second to think about what you
could have done to find it faster.  Next time you see something
similar, you will be able to find the bug more quickly.

Remember, the goal is not just to make the program
work.  The goal is to learn how to make the program work.


\chapter{Analysis of Algorithms}

\begin{quote}
This appendix is an edited excerpt from {\it Think Complexity}, by
Allen B. Downey, also published by O'Reilly Media (2011).  When you
are done with this book, you might want to move on to that one.
\end{quote}

{\bf Analysis of algorithms} is a branch of computer science that
studies the performance of algorithms, especially their run time and
space requirements.  See
\url{http://en.wikipedia.org/wiki/Analysis_of_algorithms}.
\index{algorithm} \index{analysis of algorithms}

The practical goal of algorithm analysis is to predict the performance
of different algorithms in order to guide design decisions.

During the 2008 United States Presidential Campaign, candidate
Barack Obama was asked to perform an impromptu analysis when
he visited Google.  Chief executive Eric Schmidt jokingly asked him
for ``the most efficient way to sort a million 32-bit integers.''
Obama had apparently been tipped off, because he quickly
replied, ``I think the bubble sort would be the wrong way to go.''
See \url{http://www.youtube.com/watch?v=k4RRi_ntQc8}.
\index{Obama, Barack}
\index{Schmidt, Eric}
\index{bubble sort}

This is true: bubble sort is conceptually simple but slow for
large datasets.  The answer Schmidt was probably looking for is
``radix sort'' (\url{http://en.wikipedia.org/wiki/Radix_sort})\footnote{
But if you get a question like this in an interview, I think
a better answer is, ``The fastest way to sort a million integers
is to use whatever sort function is provided by the language
I'm using.  Its performance is good enough for the vast majority
of applications, but if it turned out that my application was too
slow, I would use a profiler to see where the time was being
spent.  If it looked like a faster sort algorithm would have
a significant effect on performance, then I would look
around for a good implementation of radix sort.''}.
\index{radix sort}

The goal of algorithm analysis is to make meaningful
comparisons between algorithms, but there are some problems:
\index{comparing algorithms}

\begin{itemize}

\item The relative performance of the algorithms might
depend on characteristics of the hardware, so one algorithm
might be faster on Machine A, another on Machine B.
The general solution to this problem is to specify a
{\bf machine model} and analyze the number of steps, or
operations, an algorithm requires under a given model.
\index{machine model}

\item Relative performance might depend on the details of
the dataset.  For example, some sorting
algorithms run faster if the data are already partially sorted;
other algorithms run slower in this case.
A common way to avoid this problem is to analyze the
{\bf worst case} scenario.  It is sometimes useful to
analyze average case performance, but that's usually harder,
and it might not be obvious what set of cases to average over.
\index{worst case}
\index{average case}

\item Relative performance also depends on the size of the
problem.  A sorting algorithm that is fast for small lists
might be slow for long lists.
The usual solution to this problem is to express run time
(or number of operations) as a function of problem size,
and to compare the functions {\bf asymptotically} as the problem
size increases.
\index{asymptotic analysis}

\end{itemize}

The good thing about this kind of comparison that it lends
itself to simple classification of algorithms.  For example,
if I know that the run time of Algorithm A tends to be
proportional to the size of the input, $n$, and Algorithm B
tends to be proportional to $n^2$, then I
expect A to be faster than B for large values of $n$.

This kind of analysis comes with some caveats, but we'll get
to that later.


\section{Order of growth}

Suppose you have analyzed two algorithms and expressed
their run times in terms of the size of the input:
Algorithm A takes $100n+1$ steps to solve a problem with
size $n$; Algorithm B takes $n^2 + n + 1$ steps.
\index{order of growth}

The following table shows the run time of these algorithms
for different problem sizes:

\begin{tabular}{|r|r|r|}
\hline
Input     &   Run time of     & Run time of \\
size      &   Algorithm A     & Algorithm B \\
\hline
10        &   1 001           & 111         \\
100       &   10 001          & 10 101         \\
1 000     &   100 001         & 1 001 001         \\
10 000    &   1 000 001       & $> 10^{10}$         \\
\hline
\end{tabular}

At $n=10$, Algorithm A looks pretty bad; it takes almost 10 times
longer than Algorithm B.  But for $n=100$ they are about the same, and
for larger values A is much better.

The fundamental reason is that for large values of $n$, any function
that contains an $n^2$ term will grow faster than a function whose
leading term is $n$.  The {\bf leading term} is the term with the
highest exponent.
\index{leading term}
\index{exponent}

For Algorithm A, the leading term has a large coefficient, 100, which
is why B does better than A for small $n$.  But regardless of the
coefficients, there will always be some value of $n$ where
$a n^2 > b n$.
\index{leading coefficient}

The same argument applies to the non-leading terms.  Even if the run
time of Algorithm A were $n+1000000$, it would still be better than
Algorithm B for sufficiently large $n$.

In general, we expect an algorithm with a smaller leading term to be a
better algorithm for large problems, but for smaller problems, there
may be a {\bf crossover point} where another algorithm is better.  The
location of the crossover point depends on the details of the
algorithms, the inputs, and the hardware, so it is usually ignored for
purposes of algorithmic analysis.  But that doesn't mean you can forget
about it.
\index{crossover point}

If two algorithms have the same leading order term, it is hard to say
which is better; again, the answer depends on the details.  So for
algorithmic analysis, functions with the same leading term
are considered equivalent, even if they have different coefficients.

An {\bf order of growth} is a set of functions whose asymptotic growth
behavior is considered equivalent.  For example, $2n$, $100n$ and $n+1$
belong to the same order of growth, which is written $O(n)$ in
{\bf Big-Oh notation} and often called {\bf linear} because every function
in the set grows linearly with $n$.
\index{big-oh notation}
\index{linear growth}

All functions with the leading term $n^2$ belong to $O(n^2)$; they are
{\bf quadratic}, which is a fancy word for functions with the
leading term $n^2$.
\index{quadratic growth}

The following table shows some of the orders of growth that
appear most commonly in algorithmic analysis,
in increasing order of badness.
\index{badness}

\begin{tabular}{|r|r|r|}
\hline
Order of     &   Name      \\
growth       &               \\
\hline
$O(1)$             & constant \\
$O(\log_b n)$      & logarithmic (for any $b$) \\
$O(n)$             & linear \\
$O(n \log_b n)$    & ``en log en'' \\
$O(n^2)$           & quadratic     \\
$O(n^3)$           & cubic     \\
$O(c^n)$           & exponential (for any $c$)    \\
\hline
\end{tabular}

For the logarithmic terms, the base of the logarithm doesn't matter;
changing bases is the equivalent of multiplying by a constant, which
doesn't change the order of growth.  Similarly, all exponential
functions belong to the same order of growth regardless of the base of
the exponent.
Exponential functions grow very quickly, so exponential algorithms are
only useful for small problems.
\index{logarithmic growth}
\index{exponential growth}


\begin{exercise}

Read the Wikipedia page on Big-Oh notation at
\url{http://en.wikipedia.org/wiki/Big_O_notation} and
answer the following questions:

\begin{enumerate}
\item What is the order of growth of $n^3 + n^2$?
What about $1000000 n^3 + n^2$?
What about $n^3 + 1000000 n^2$?

\item What is the order of growth of $(n^2 + n) \cdot (n + 1)$?  Before
  you start multiplying, remember that you only need the leading term.

\item If $f$ is in $O(g)$, for some unspecified function $g$, what can
  we say about $af+b$?

\item If $f_1$ and $f_2$ are in $O(g)$, what can we say about $f_1 + f_2$?

\item If  $f_1$ is in $O(g)$
and $f_2$ is in $O(h)$,
what can we say about  $f_1 + f_2$?

\item If  $f_1$ is in $O(g)$ and $f_2$ is $O(h)$,
what can we say about  $f_1 \cdot f_2$?
\end{enumerate}

\end{exercise}

Programmers who care about performance often find this kind of
analysis hard to swallow.  They have a point: sometimes the
coefficients and the non-leading terms make a real difference.
Sometimes the details of the hardware, the programming language, and
the characteristics of the input make a big difference.  And for small
problems asymptotic behavior is irrelevant.
\index{practical analysis of algorithms}

But if you keep those caveats in mind, algorithmic analysis is a
useful tool.  At least for large problems, the ``better'' algorithms
is usually better, and sometimes it is {\em much} better.  The
difference between two algorithms with the same order of growth is
usually a constant factor, but the difference between a good algorithm
and a bad algorithm is unbounded!
\index{unbounded}


\section{Analysis of basic Python operations}

Most arithmetic operations are constant time; multiplication
usually takes longer than addition and subtraction, and division
takes even longer, but these run times don't
depend on the magnitude of the operands.  Very large integers
are an exception; in that case the run time increases
with the number of digits.
\index{analysis of primitives}

Indexing operations---reading or writing elements in a sequence
or dictionary---are also constant time, regardless of the size
of the data structure.
\index{indexing}

A {\tt for} loop that traverses a sequence or dictionary is
usually linear, as long as all of the operations in the body
of the loop are constant time.  For example, adding up the
elements of a list is linear:

\begin{verbatim}
    total = 0
    for x in t:
        total += x
\end{verbatim}

The built-in function {\tt sum} is also linear because it does
the same thing, but it tends to be faster because it is a more
efficient implementation; in the language of algorithmic analysis,
it has a smaller leading coefficient.

If you use the same loop to ``add'' a list of strings, the
run time is quadratic
because string concatenation is linear.
\index{string concatenation}

The string method {\tt join} is usually faster because it is
linear in the total length of the strings.
\index{join@{\tt join}}

As a rule of thumb, if the body of a loop is in $O(n^a)$ then
the whole loop is in $O(n^{a+1})$.  The exception is if you can
show that the loop exits after a constant number of iterations.
If a loop runs $k$ times regardless of $n$, then
the loop is in $O(n^a)$, even for large $k$.

Multiplying by $k$ doesn't change the order of growth, but neither
does dividing.  So if the body of a loop is in $O(n^a)$ and it runs
$n/k$ times, the loop is in $O(n^{a+1})$, even for large $k$.

Most string and tuple operations are linear, except indexing and {\tt
  len}, which are constant time.  The built-in functions {\tt min} and
{\tt max} are linear.  The run-time of a slice operation is
proportional to the length of the output, but independent of the size
of the input.
\index{string methods}
\index{tuple methods}

All string methods are linear, but if the lengths of
the strings are bounded by a constant---for example, operations on single
characters---they are considered constant time.

Most list methods are linear, but there are some exceptions:
\index{list methods}

\begin{itemize}

\item Adding an element to the end of a list is constant time on
average; when it runs out of room it occasionally gets copied
to a bigger location, but the total time for $n$ operations
is $O(n)$, so we say that the ``amortized'' time for one
operation is $O(1)$.

\item Removing an element from the end of a list is constant time.

\item Sorting is $O(n \log n)$.
\index{sorting}

\end{itemize}

Most dictionary operations and methods are constant time, but
there are some exceptions:
\index{dictionary methods}

\begin{itemize}

\item The run time of {\tt copy} is proportional to the number of
  elements, but not the size of the elements (it copies references,
  not the elements themselves).

\item The run time of {\tt update} is
  proportional to the size of the dictionary passed as a parameter,
  not the dictionary being updated.

\item {\tt keys}, {\tt values} and {\tt items} are linear because they
  return new lists; {\tt iterkeys}, {\tt itervalues} and {\tt
    iteritems} are constant time because they return iterators.  But
  if you loop through the iterators, the loop will be linear.  Using
  the ``iter'' functions saves some overhead, but it doesn't change
  the order of growth unless the number of items you access is
  bounded.

\end{itemize}

The performance of dictionaries is one of the minor miracles of
computer science.  We will see how they work in
Section~\ref{hashtable}.


\begin{exercise}

Read the Wikipedia page on sorting algorithms at
\url{http://en.wikipedia.org/wiki/Sorting_algorithm} and answer
the following questions:
\index{sorting}

\begin{enumerate}

\item What is a ``comparison sort?'' What is the best worst-case order
  of growth for a comparison sort?  What is the best worst-case order
  of growth for any sort algorithm?
\index{comparison sort}

\item What is the order of growth of bubble sort, and why does Barack
  Obama think it is ``the wrong way to go?''

\item What is the order of growth of radix sort?  What preconditions
  do we need to use it?

\item What is a stable sort and why might it matter in practice?
\index{stable sort}

\item What is the worst sorting algorithm (that has a name)?

\item What sort algorithm does the C library use?  What sort algorithm
  does Python use?  Are these algorithms stable?  You might have to
  Google around to find these answers.

\item Many of the non-comparison sorts are linear, so why does does
  Python use an $O(n \log n)$ comparison sort?

\end{enumerate}

\end{exercise}


\section{Analysis of search algorithms}

A {\bf search} is an algorithm that takes a collection and a target
item and determines whether the target is in the collection, often
returning the index of the target.
\index{search}

The simplest search algorithm is a ``linear search,'' which traverses
the items of the collection in order, stopping if it finds the target.
In the worst case it has to traverse the entire collection, so the run
time is linear.
\index{linear search}

The {\tt in} operator for sequences uses a linear search; so do string
methods like {\tt find} and {\tt count}.
\index{in@{\tt in} operator}

If the elements of the sequence are in order, you can use a {\bf
  bisection search}, which is $O(\log n)$.  Bisection search is
similar to the algorithm you probably use to look a word up in a
dictionary (a real dictionary, not the data structure).  Instead of starting at
the beginning and checking each item in order, you start with the item
in the middle and check whether the word you are looking for comes
before or after.  If it comes before, then you search the first half
of the sequence.  Otherwise you search the second half.  Either way,
you cut the number of remaining items in half.  \index{bisection
  search}

If the sequence has 1,000,000 items, it will take about 20 steps to
find the word or conclude that it's not there.  So that's about 50,000
times faster than a linear search.

\begin{exercise}

Write a function called {\tt bisection} that takes a sorted list
and a target value and returns the index of the value
in the list, if it's there, or {\tt None} if it's not.
\index{bisect@{\tt bisect} module}

\index{bisect module}
\index{module!bisect}

Or you could read the documentation of the {\tt bisect} module
and use that!

\end{exercise}

Bisection search can be much faster than linear search, but
it requires the sequence to be in order, which might require
extra work.

There is another data structure, called a {\bf hashtable} that
is even faster---it can do a search in constant time---and it
doesn't require the items to be sorted.  Python dictionaries
are implemented using hashtables, which is why most dictionary
operations, including the {\tt in} operator, are constant time.


\section{Hashtables}
\label{hashtable}

To explain how hashtables work and why their performance is so
good, I start with a simple implementation of a map and
gradually improve it until it's a hashtable.
\index{hashtable}

I use Python to demonstrate these implementations, but in real
life you wouldn't write code like this in Python; you would just use a
dictionary!  So for the rest of this chapter, you have to imagine that
dictionaries don't exist and you want to implement a data structure
that maps from keys to values.  The operations you have to
implement are:

\begin{description}

\item[{\tt add(k, v)}:] Add a new item that maps from key {\tt k}
to value {\tt v}.  With a Python dictionary, {\tt d}, this operation
is written {\tt d[k] = v}.

\item[{\tt get(target)}:] Look up and return the value that corresponds
to key {\tt target}.  With a Python dictionary, {\tt d}, this operation
is written {\tt d[target]} or {\tt d.get(target)}.

\end{description}

For now, I assume that each key only appears once.
The simplest implementation of this interface uses a list of
tuples, where each tuple is a key-value pair.
\index{LinearMap@{\tt LinearMap}}

\begin{verbatim}
class LinearMap(object):

    def __init__(self):
        self.items = []

    def add(self, k, v):
        self.items.append((k, v))

    def get(self, k):
        for key, val in self.items:
            if key == k:
                return val
        raise KeyError
\end{verbatim}

{\tt add} appends a key-value tuple to the list of items, which
takes constant time.

{\tt get} uses a {\tt for} loop to search the list:
if it finds the target key it returns the corresponding value;
otherwise it raises a {\tt KeyError}.
So {\tt get} is linear.
\index{KeyError@{\tt KeyError}}

An alternative is to keep the list sorted by key.  Then {\tt get}
could use a bisection search, which is $O(\log n)$.  But inserting a
new item in the middle of a list is linear, so this might not be the
best option.  There are other data structures (see
  \url{http://en.wikipedia.org/wiki/Red-black_tree})  that can implement {\tt
  add} and {\tt get} in log time, but that's still not as good as
constant time, so let's move on.
\index{red-black tree}

One way to improve {\tt LinearMap} is to break the list of key-value
pairs into smaller lists.  Here's an implementation called
{\tt BetterMap}, which is a list of 100 LinearMaps.  As we'll see
in a second, the order of growth for {\tt get} is still linear,
but {\tt BetterMap} is a step on the path toward hashtables:
\index{BetterMap@{\tt BetterMap}}

\begin{verbatim}
class BetterMap(object):

    def __init__(self, n=100):
        self.maps = []
        for i in range(n):
            self.maps.append(LinearMap())

    def find_map(self, k):
        index = hash(k) % len(self.maps)
        return self.maps[index]

    def add(self, k, v):
        m = self.find_map(k)
        m.add(k, v)

    def get(self, k):
        m = self.find_map(k)
        return m.get(k)
\end{verbatim}

\verb"__init__" makes a list of {\tt n} {\tt LinearMap}s.

\verb"find_map" is used by
{\tt add} and {\tt get}
to figure out which map to put the
new item in, or which map to search.

\verb"find_map" uses the built-in function {\tt hash}, which takes
almost any Python object and returns an integer.  A limitation of this
implementation is that it only works with hashable keys.  Mutable
types like lists and dictionaries are unhashable.
\index{hash function}

Hashable objects that are considered equal return the same hash value,
but the converse is not necessarily true: two different objects
can return the same hash value.

\verb"find_map" uses the modulus operator to wrap the hash values
into the range from 0 to {\tt len(self.maps)}, so the result is a legal
index into the list.  Of course, this means that many different
hash values will wrap onto the same index.  But if the hash function
spreads things out pretty evenly (which is what hash functions
are designed to do), then we expect $n/100$ items per LinearMap.

Since the run time of {\tt LinearMap.get} is proportional to the
number of items, we expect BetterMap to be about 100 times faster
than LinearMap.  The order of growth is still linear, but the
leading coefficient is smaller.  That's nice, but still not
as good as a hashtable.

Here (finally) is the crucial idea that makes hashtables fast: if you
can keep the maximum length of the LinearMaps bounded, {\tt
  LinearMap.get} is constant time.  All you have to do is keep track
of the number of items and when the number of
items per LinearMap exceeds a threshold, resize the hashtable by
adding more LinearMaps.
\index{bounded}

Here is an implementation of a hashtable:
\index{HashMap}

\begin{verbatim}
class HashMap(object):

    def __init__(self):
        self.maps = BetterMap(2)
        self.num = 0

    def get(self, k):
        return self.maps.get(k)

    def add(self, k, v):
        if self.num == len(self.maps.maps):
            self.resize()

        self.maps.add(k, v)
        self.num += 1

    def resize(self):
        new_maps = BetterMap(self.num * 2)

        for m in self.maps.maps:
            for k, v in m.items:
                new_maps.add(k, v)

        self.maps = new_maps
\end{verbatim}

Each {\tt HashMap} contains a {\tt BetterMap}; \verb"__init__" starts
with just 2 LinearMaps and initializes {\tt num}, which keeps track of
the number of items.

{\tt get} just dispatches to {\tt BetterMap}.  The real work happens
in {\tt add}, which checks the number of items and the size of the
{\tt BetterMap}: if they are equal, the average number of items per
LinearMap is 1, so it calls {\tt resize}.

{\tt resize} make a new {\tt BetterMap}, twice as big as the previous
one, and then ``rehashes'' the items from the old map to the new.

Rehashing is necessary because changing the number of LinearMaps
changes the denominator of the modulus operator in
\verb"find_map".  That means that some objects that used
to wrap into the same LinearMap will get split up (which is
what we wanted, right?).
\index{rehashing}

Rehashing is linear, so
{\tt resize} is linear, which might seem bad, since I promised
that {\tt add} would be constant time.  But remember that
we don't have to resize every time, so {\tt add} is usually
constant time and only occasionally linear.  The total amount
of work to run {\tt add} $n$ times is proportional to $n$,
so the average time of each {\tt add} is constant time!
\index{constant time}

To see how this works, think about starting with an empty
HashTable and adding a sequence of items.  We start with 2 LinearMaps,
so the first 2 adds are fast (no resizing required).  Let's
say that they take one unit of work each.  The next add
requires a resize, so we have to rehash the first two
items (let's call that 2 more units of work) and then
add the third item (one more unit).  Adding the next item
costs 1 unit, so the total so far is
6 units of work for 4 items.

The next {\tt add} costs 5 units, but the next three
are only one unit each, so the total is 14 units for the
first 8 adds.

The next {\tt add} costs 9 units, but then we can add 7 more
before the next resize, so the total is 30 units for the
first 16 adds.

After 32 adds, the total cost is 62 units, and I hope you are starting
to see a pattern.  After $n$ adds, where $n$ is a power of two, the
total cost is $2n-2$ units, so the average work per add is
a little less than 2 units.  When $n$ is a power of two, that's
the best case; for other values of $n$ the average work is a little
higher, but that's not important.  The important thing is that it
is $O(1)$.
\index{average cost}

Figure~\ref{fig.hash} shows how this works graphically.  Each
block represents a unit of work.  The columns show the total
work for each add in order from left to right: the first two
{\tt adds} cost 1 units, the third costs 3 units, etc.

\begin{figure}
\centerline{\includegraphics[scale=1.0]{figs/towers.pdf}}
\caption{The cost of a hashtable add.\label{fig.hash}}
\end{figure}

The extra work of rehashing appears as a sequence of increasingly
tall towers with increasing space between them.  Now if you knock
over the towers, amortizing the cost of resizing over all
adds, you can see graphically that the total cost after $n$
adds is $2n - 2$.

An important feature of this algorithm is that when we resize the
HashTable it grows geometrically; that is, we multiply the size by a
constant.  If you increase the size
arithmetically---adding a fixed number each time---the average time
per {\tt add} is linear.
\index{geometric resizing}

You can download my implementation of HashMap from
\url{http://thinkpython/code/Map.py}, but remember that there
is no reason to use it; if you want a map, just use a Python dictionary.






\chapter{Lumpy}
\label{lumpy}

Throughout the book, I have used diagrams to represent the state of
running programs.
\index{Lumpy}

In Section~\ref{variables}, we used a state diagram to show the names
and values of variables.  In Section~\ref{stackdiagram} I introduced a
stack diagram, which shows one frame for each function call.  Each
frame shows the parameters and local variables for the function or
method.  Stack diagrams for recursive functions appear in
Section~\ref{recursive.stack} and Section~\ref{more.recursion}.
\index{stack diagram} \index{diagram!stack}
\index{state diagram} \index{diagram!state}

Section~\ref{mutable} shows what a list looks like in a state diagram,
Section~\ref{invert} shows what a dictionary looks like, and
Section~\ref{dictuple} shows two ways to represent tuples.

Section~\ref{attributes} introduces object diagrams, which show the
state of an object's attributes, and their attributes, and so on.
Section~\ref{rectangles} has object diagrams for Rectangles and
their embedded Points.  Section~\ref{time.object} shows the state
of a Time object.
Section~\ref{class.attribute} has a diagram that includes a class
object and an instance, each with their own attributes.
\index{object diagram}
\index{diagram!object}

Finally, Section~\ref{class.diagram} introduces class diagrams,
which show the classes that make up a program and the relationships
between them.
\index{class diagram}
\index{diagram!class}

These diagrams are based on the Unified Modeling Language (UML), which
is a standardized graphical language used by software engineers
to communicate about program design, especially for object-oriented
programs.
\index{Unified Modeling Language}
\index{UML}

UML is a rich language with many kinds of diagrams that represent
many kinds of relationship between objects and classes.  What I presented
in this book is a small subset of the language, but it is the subset
most commonly used in practice.

The purpose of this appendix is to review the diagrams presented in
the previous chapters, and to introduce Lumpy.  Lumpy, which stands
for ``UML in Python,'' with some of the letters rearranged, is part of
Swampy, which you already installed if you worked on the case study in
Chapter~\ref{turtlechap} or Chapter~\ref{tkinter}, or if you did
Exercise~\ref{canvas},
\index{Lumpy}
\index{Swampy}

Lumpy uses Python's {\tt inspect} module to examine the state of a running
program and generate object diagrams (including stack diagrams) and
class diagrams.

\section{State diagram}

\begin{figure}
\centerline
{\includegraphics[scale=0.7]{figs/lumpydemo1.pdf}}
\caption{State diagram generated by Lumpy.}
\label{fig.lumpy1}
\end{figure}

Here's an example that uses Lumpy to generate a state diagram.
\index{state diagram} \index{diagram!state}

\begin{verbatim}
from swampy.Lumpy import Lumpy

lumpy = Lumpy()
lumpy.make_reference()

message = 'And now for something completely different'
n = 17
pi = 3.1415926535897932

lumpy.object_diagram()
\end{verbatim}

The first line imports the Lumpy class from {\tt swampy.Lumpy}.
If you don't have Swampy installed as a package, make sure
the Swampy files are in Python's search path and use this
{\tt import} statement instead:

\begin{verbatim}
from Lumpy import Lumpy
\end{verbatim}

The next lines create a {\tt Lumpy} object and make a ``reference''
point, which means that Lumpy records the objects that have been
defined so far.

Next we define new variables and invoke \verb"object_diagram",
which draws the objects that have been defined since the reference
point, in this case {\tt message}, {\tt n} and {\tt pi}.

Figure~\ref{fig.lumpy1} shows the result.  The graphical style is
different from what I showed earlier; for example, each
reference is represented by a circle next to the variable name and a
line to the value.  And long strings are truncated.  But the
information conveyed by the diagram is the same.

The variable names are in a frame labeled \verb"<module>", which
indicates that these are module-level variables, also known as
global.
\index{global variable}
\index{variable!global}
\index{module-level variable}
\index{variable!module-level}

You can download this example from
\url{http://thinkpython.com/code/lumpy_demo1.py}.  Try adding some
additional assignments and see what the diagram looks like.


\section{Stack diagram}

\begin{figure}
\centerline
{\includegraphics[scale=0.7]{figs/lumpydemo2.pdf}}
\caption{Stack diagram.}
\label{fig.lumpy2}
\end{figure}

Here's an example that uses Lumpy to generate a stack diagram.
You can download it from \url{http://thinkpython.com/code/lumpy_demo2.py}.
\index{stack diagram} \index{diagram!stack}

\begin{verbatim}
from swampy.Lumpy import Lumpy

def countdown(n):
    if n <= 0:
        print 'Blastoff!'
        lumpy.object_diagram()
    else:
        print n
        countdown(n-1)

lumpy = Lumpy()
lumpy.make_reference()
countdown(3)
\end{verbatim}

Figure~\ref{fig.lumpy2} shows the result.  Each frame is represented
with a box that has the function's name outside and variables inside.
Since this function is recursive, there is one frame for each
level of recursion.
\index{recursion}
\index{function frame}
\index{frame}

Remember that a stack diagram shows the state of the program at
a particular point in its execution.  To get the diagram you want,
sometimes you have to think about where to invoke \verb"object_diagram".

In this case I invoke \verb"object_diagram" after executing the base
case of the recursion; that way the stack diagram shows each level of
the recursion.  You can call \verb"object_diagram" more than once to
get a series of snapshots of the program's execution.
\index{base case}


\section{Object diagrams}

\begin{figure}
\centerline
{\includegraphics[scale=0.7]{figs/lumpydemo3.pdf}}
\caption{Object diagram.}
\label{fig.lumpy3}
\end{figure}

This example generates an object diagram showing the lists from
Section~\ref{sequence}.  You can download it from
\url{http://thinkpython.com/code/lumpy_demo3.py}.
\index{object diagram} \index{diagram!object}

\begin{verbatim}
from swampy.Lumpy import Lumpy

lumpy = Lumpy()
lumpy.make_reference()

cheeses = ['Cheddar', 'Edam', 'Gouda']
numbers = [17, 123]
empty = []

lumpy.object_diagram()
\end{verbatim}

Figure~\ref{fig.lumpy3} shows the result.  Lists are represented by
a box that shows the indices mapping to the elements.  This representation
is slightly misleading, since indices are not actually
part of the list, but I think they make the diagram easier to
read.  The empty list is represented by an empty box.
\index{list index}

\begin{figure}
\centerline
{\includegraphics[scale=0.7]{figs/lumpydemo4.pdf}}
\caption{Object diagram.}
\label{fig.lumpy4}
\end{figure}

And here's an example
showing the dictionaries from Section~\ref{invert}.  You can download
it from \url{http://thinkpython.com/code/lumpy_demo4.py}.
\index{dictionary}

\begin{verbatim}
from swampy.Lumpy import Lumpy

lumpy = Lumpy()
lumpy.make_reference()

hist = histogram('parrot')
inverse = invert_dict(hist)

lumpy.object_diagram()
\end{verbatim}

Figure~\ref{fig.lumpy4} shows the result.  {\tt hist} is a dictionary
that maps from characters (single-letter strings) to integers;
{\tt inverse} maps from integers to lists of strings.

\begin{figure}
\centerline
{\includegraphics[scale=0.7]{figs/lumpydemo5.pdf}}
\caption{Object diagram.}
\label{fig.lumpy5}
\end{figure}

This example generates an object diagram for Point and Rectangle
objects, as in Section~\ref{copying}.  You can download it from
\url{http://thinkpython.com/code/lumpy_demo5.py}.
\index{Point class}
\index{class!Point}
\index{Rectangle class}
\index{class!Rectangle}

\begin{verbatim}
import copy
from swampy.Lumpy import Lumpy

lumpy = Lumpy()
lumpy.make_reference()

box = Rectangle()
box.width = 100.0
box.height = 200.0
box.corner = Point()
box.corner.x = 0.0
box.corner.y = 0.0

box2 = copy.copy(box)

lumpy.object_diagram()
\end{verbatim}

Figure~\ref{fig.lumpy5} shows the result.  {\tt copy.copy} make a
shallow copy, so {\tt box} and {\tt box2} have their own {\tt width}
and {\tt height}, but they share the same embedded Point object.  This
kind of sharing is usually fine with immutable objects, but with
mutable types, it is highly error-prone.
\index{copy}
\index{shallow copy}

\section{Function and class objects}

\begin{figure}
\centerline
{\includegraphics[scale=0.7]{figs/lumpydemo6.pdf}}
\caption{Object diagram.}
\label{fig.lumpy6}
\end{figure}

When I use Lumpy to make object diagrams, I usually define the functions
and classes before I make the reference point.  That way, function
and class objects don't appear in the diagram.
\index{function object}
\index{object!function}
\index{class object}
\index{object!class}

But if you are passing functions and classes as parameters, you might
want them to appear.  This example shows what that looks like;
you can download it from
\url{http://thinkpython.com/code/lumpy_demo6.py}.

\begin{verbatim}
import copy
from swampy.Lumpy import Lumpy

lumpy = Lumpy()
lumpy.make_reference()

class Point(object):
    """Represents a point in 2-D space."""

class Rectangle(object):
    """Represents a rectangle."""

def instantiate(constructor):
    """Instantiates a new object."""
    obj = constructor()
    lumpy.object_diagram()
    return obj

point = instantiate(Point)
\end{verbatim}

Figure~\ref{fig.lumpy6} shows the result.  Since we invoke
\verb"object_diagram" inside a function, we get a stack diagram
with a frame for the module-level variables and for the invocation
of {\tt instantiate}.

At the module level, {\tt Point} and {\tt Rectangle} refer to
class objects (which have type {\tt type}); {\tt instantiate}
refers to a function object.
\index{instantiate}
\index{constructor}

This diagram might clarify two points of common confusion: (1) the
difference between the class object, {\tt Point}, and the instance of
Point, {\tt obj}, and (2) the difference between the function object
created when {\tt instantiate} is defined, and the frame created with
it is called.


\section{Class Diagrams}

\begin{figure}
\centerline
{\includegraphics[scale=0.7]{figs/lumpydemo7.pdf}}
\caption{Class diagram.}
\label{fig.lumpy7}
\end{figure}

\begin{figure}
\centerline
{\includegraphics[scale=0.7]{figs/lumpydemo8.pdf}}
\caption{Class diagram.}
\label{fig.lumpy8}
\end{figure}

Although I distinguish between state diagrams, stack diagrams and
object diagrams, they are mostly the same thing: they show the
state of a running program at a point in time.
\index{class diagram}
\index{diagram!class}

Class diagrams are different.  They show the classes that make up a
program and the relationships between them.  They are timeless in the
sense that they describe the program as a whole, not any particular
point in time.  For example, if an instance of Class A generally
contains a reference to an instance of Class B, we say there is a
``HAS-A relationship'' between those classes.
\index{HAS-A relationship}
\index{class diagram}
\index{diagram!class}
\index{UML}

Here's an example that shows a HAS-A relationship.  You can download
it from \url{http://thinkpython.com/code/lumpy_demo7.py}.

\begin{verbatim}
from swampy.Lumpy import Lumpy

lumpy = Lumpy()
lumpy.make_reference()

box = Rectangle()
box.width = 100.0
box.height = 200.0
box.corner = Point()
box.corner.x = 0.0
box.corner.y = 0.0

lumpy.class_diagram()
\end{verbatim}

Figure~\ref{fig.lumpy7} shows the result.
Each class is represented with a box that contains the name of the
class, any methods the class provides, any class variables, and
any instance variables.  In this example, {\tt Rectangle} and {\tt Point}
have instance variables, but no methods or class variables.

The arrow from {\tt Rectangle} to {\tt Point} shows that Rectangles
contain an embedded Point.  In addition, {\tt Rectangle} and {\tt
  Point} both inherit from {\tt object}, which is represented in
the diagram with a triangle-headed arrow.
\index{IS-A relationship}

Here's a more complex example using my solution to Exercise~\ref{poker}.
You can download
the code from \url{http://thinkpython.com/code/lumpy_demo8.py};
you will also need \url{http://thinkpython.com/code/PokerHand.py}.

\begin{verbatim}
from swampy.Lumpy import Lumpy

from PokerHand import *

lumpy = Lumpy()
lumpy.make_reference()

deck = Deck()
hand = PokerHand()
deck.move_cards(hand, 7)

lumpy.class_diagram()
\end{verbatim}

Figure~\ref{fig.lumpy8} shows the result.
{\tt PokerHand} inherits from {\tt Hand}, which inherits from {\tt Deck}.
Both {\tt Deck} and {\tt PokerHand} have Cards.
\index{Card class}
\index{Deck class}
\index{Hand class}

This diagram does not show that {\tt Hand} also has cards, because
in the program there are no instances of Hand.  This example
demonstrates a limitation of Lumpy; it only knows about the
attributes and HAS-A relationships of objects that are instantiated.

\printindex

\clearemptydoublepage
%\blankpage
%\blankpage
%\blankpage


\end{document}
